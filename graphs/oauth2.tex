\documentclass[crop,tikz]{standalone}% 'crop' is the default for v1.0, before it was 'preview'
%\usetikzlibrary{...}% tikz package already loaded by 'tikz' option

\usepackage[ngerman]{babel}
\usepackage{tikz-uml}

\begin{document}
\begin{tikzpicture}[scale=\textwidth/15.2cm,samples=200]
	\tikzumlset{fill object = white, fill call = gray!20} 
	\tikzstyle{every node}=[font=\small]

	\begin{umlseqdiag}
		\umlactor[no ddots]{Resource Owner}
		\umlobject[no ddots]{User Agent}
		\umlobject[no ddots]{Web Service}
		\umlobject[no ddots]{Authorization Server}
		\umlobject[no ddots]{Resource Server}

		\begin{umlcall}[dt=8, op={Anfrage Resource}, return={Anzeige Authorization GUI}]{Resource Owner}{User Agent}

			\begin{umlcall}[dt=8, op={GET Resource}, return={302 Authorization Reqeust}]{User Agent}{Web Service}
			\end{umlcall}

			\begin{umlcall}[dt=8, op={GET Authorization Request}, return={Authorization GUI}]{User Agent}{Authorization Server}
			\end{umlcall}
		\end{umlcall}


		\begin{umlcall}[dt=8, op={Eingabe Authorization}, return={Anzeige Resource}]{Resource Owner}{User Agent}

			\begin{umlcall}[dt=8, op={POST Bestätigung und Authorization}, return={302 Authorization Code}]{User Agent}{Authorization Server}
			\end{umlcall}

			\begin{umlcall}[dt=8, op={Authorization Code}, return={200 Resource}]{User Agent}{Web Service}


				\begin{umlcall}[dt=8, op={Authorization Code and Redirection URL}, return={Access and Refresh Token}]{Web Service}{Authorization Server}
				\end{umlcall}
				
				\begin{umlcall}[dt=8, op={GET Request mit Authorization Code}, return={200 Resource}]{Web Service}{Resource Server}
				\end{umlcall}

			\end{umlcall}
		\end{umlcall}
	\end{umlseqdiag}
\end{tikzpicture}
\end{document}
