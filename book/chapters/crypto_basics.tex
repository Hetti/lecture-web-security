\chapter{Kryptographische Grundlagen}

Kryptographie beschreibt die Technik (und Kunst) über nicht-vertrauenswürdige Kanäle bzw. Speicherorte Daten integritäts- und vertraulichkeitsgesichert zu übertragen. Dadurch kann Kryptographie als Mittel gegen \textit{spoofing}, \textit{tampering}, \textit{repudiation} und \textit{information disclosure} dienen. Dieses Kapitel soll eine (extrem) kurze Einführung in die, in diesem Dokument, verwendeten Konzepte geben.

Bei der Verschlüsselung wird der ursprüngliche Text (häufig \textit{plaintext} genannt) durch den Algorithmus in einen neuen, verschlüsselten, Text (häufig \textit{ciphertext} genannt) konvertiert. Dieser Ciphertext kann durch die Entschlüsselung wieder in den Plaintext zurück verwandelt werden. Die hierbei verwendeten Algorithmen werden häufig \textit{Cipher} genannt. In der Literatur werden die dabei beteiligten Partein meistens ident benannt: \textit{Alice} und \textit{Bob} sind die beiden Parteien die miteinander sicher kommunizieren wollen. \textit{Eve} ist ein Angreifer, der diese Nachrichten abhören, aber nicht modifizieren kann; \textit{Mallory} ist ein Angreifer, der auch aktiv angreifen darf.

Grundlegend sollten folgende Grundsätze bei der Verwendung von Kryptographie beachtet werden:

\begin{itemize}
	\item Niemals selbst ein kryptographisches System entwerfen, sondern immer ein etabliertes (und getestetes) System verwenden.
	\item Niemals selbst einen kryptographischen Algorithmus/Bibliothek implementieren, sondern immer etablierte und getestet Komponenten verwenden.
	\item Die richtige kryptographische Methode wählen.
	\item Immer davon ausgehen, dass der eigene Source Code früher oder später öffentlich wird. Aus diesem Grund darf ein kryptographischer Schlüssel (oder auch Credentials) niemals Teil des Source Codes werden.
	\item Essentiell zur sicheren Verwendung der verschiedenen kryptographischen Methoden sind die dabei verwendeten Schüssel. Diese müssen sowohl sicher gespeichert als auch transportiert werden. Noch komplexer ist das Herstellen eines Vertrauensverhältnis (Trust) zwischen den jeweiligen Kommunikationspartnern: woher weiss ein Partner, dass ein vorhandener Schlüssel eines anderen Kommunikationsparters vertrauenswürdig ist? Key Management ist komplex und sollte nicht unterschätzt werden!
\end{itemize}

Jede implementierte und konfigurierbare kryptographische Methode erhöht potentiell die Angriffsoberfläche. Ein Beispiel hierfür ist z. B. die OpenSSL-Bibliothek die dutzende Algorithmen implementiert. Als Alternative sind in den letzten Jahren kryptographische Bibliotheken wie \textit{NaCl} (``salt'') entstanden, die für jede kryptographische Methode genau eine sichere Implementierung anbieten. Auf diese Weise sollen Selektionsfehler durch Entwickler vermieden werden.

Ein häufiger verwendeter Begriff ist \textit{Rubber Hose Cryptography}. Ein noch so technisch sicheres kryptographisches System kann durch bezahlte Schläger mit einem Gummischlauch und der Androhung von Gewalt, falls das Opfer nicht den privaten Schlüssel mitteilt, günstig gebrochen werden. Anstatt durch Androhung von Gewalt kann \textit{Rubber Hose Cryptography} auch auf andere Aspekte eines Schlüsselträgers abzielen: Geld, Ideologie, Coersion oder Ego (Sex sells).

\section{Verschlüsselung}

Zur Wahrung der Vertraulichkeit von Daten wird Verschlüsselung eingesetzt. Bei dieser wird der Originaltext (engl. \textit{plaintext}) in einen verschlüsselten Text (engl. \textit{ciphertext}) konvertiert. Dieser kann wieder durch den Entschlüsselungs-Vorgang in den Originaltext zurück verwandelt werden. Verschlüsselungsalgorithmen können in zwei Familien eingeteilt werden: symmetrisch und asymmetrisch (auch public-key encryption genannt). Bei symmetrischer Verschlüsselung wird zum ver- und entschlüsseln der idente Schlüssel verwendet. Problematisch hierbei ist, dass dieser geteilte geheime Schlüssel initial zwischen allen Beteiligten verteilt werden muss. Bei der asymmetrischen Verschlüsselung wird statt einem geteilten Schlüssel ein Schlüsselpaar\footnote{Das Schlüsselpaar ist mathematisch ``verwandt''.} verwendet. Dieses besteht aus einem öffentlichen Schlüssel der zur Verschlüsselung dient und einem zugehörigen privaten Schlüssel der zum Entschlüsseln verwendet wird. Dadurch wird die Problematik des initialen Schlüsselverteilens entschärft, da nur öffentliche Schlüssel verteilt werden müssen (diese dürfen veröffentlicht bzw. verloren werden). Ein Nachteil asymmetrischer Verschlüsselung gegenüber symmetrischer Verschlüsselung ist, dass sie langsamer als symmetrische Verschlüsselung ist.

\section{Block- und Stream-Cipher}

Eine weitere Unterscheidungsmöglichkeit für Verschlüsselungsalgorithmen ist die in \textit{block} und \textit{stream} ciphers. Bei Blockciphern werden zuerst Daten angehäuft (``ein Block'' an plain-data) und dann dieser Block verschlüsselt. Bei einem Streamcipher wird jedes Zeichen sofort verschlüsselt, das Sammeln von Blocken wird so vermieden. Während Stream-Ciphers teilweise einfacher für Programmierer in ihrer Verwendung sind, werden aus Effizienzgründen fast ausschließlich Blockcipher verwendet. Werden zwei idente Blöcke mit dem identen Schlüssel verschlüsselt, würden idente verschlüsselte Blöcke entstehen. Dies erlaubt es einem Angreifer, strukturelle Informationen aus verschlüsselten Dokumenten zu extrahieren. Um dies zu vermeiden werden so genannte \textit{Block Modes} verwendet um sicherzustellen, dass idente plain-text Blöcke unterschiedliche cipher-text Blöcke produzieren. Bei Auswahl des Block Modes sollten GCM-Modes (bzw. AEAD-Varianten) bevorzugt und ECB bzw. CBC Modes vermieden werden.

\section{Integritätsschutz}

Verschlüsselung gewährleistet nicht automatisch die Integrität der verschlüsselten Daten. Hierfür müssen eigene Algorithmen verwendet werden. Häufig vorgefunden werden Hashes, Message Authentication Codes (MACs) und Signaturen. Vereinfacht ausgedruckt berechnen Hashes ausgehend von beliebig langen Eingangsdaten eine Checksumme konstanter Größe. Wird ein Hash auf identen Eingangsdaten angewandt, wird auch ein identer Hash berechnet. Ein Hash ist eine Einwegfunktion: während der zugehörige Hash zu einem Eingangsdatum schnell berechnet werden kann (gegeben den ursprünglichen Daten), ist das Berechnen der Eingangsdaten ausgehend von einem Hash realistisch nicht möglich.

Bei einem Message Authentication Code (MAC) wird der Hash um ein geheimes geteiltes Passwort erweitert. Zur Berechnung bzw. Validierung eines MACs wird dieses Passwort benötigt. Analog zur symmetrischen Verschlüsselung ergibt sich hier die Problematik der Schlüsselverteilung. Signaturen lösen dieses Problem indem sie asymmetrische (public-key) Verschlüsselung einsetzen. Bei ihnen kann die Checksumme (Signature) mit Hilfe des privaten Schlüssels erstellt und mit Hilfe des öffentlichen Schlüssels verifiziert werden. Dadurch entfällt das Problem der Schlüsselverteilung, allerdings wird auch hier der Vorteil durch geringere Geschwindigkeit erkauft.

Je nach Einsatzbereich muss nun ein geeignetes Verfahren zur Integritätssicherung und Verschlüsselung gewählt werden. Werden Daten über ein öffentliches bzw. feindliches Netzwerk transferiert ist z. B. der Einsatz eines Hashes problematisch. Falls ein Angreifer einen Datensatz abfangen und modifizieren kann, kann er ebenso einen neuen Hash berechnen und so den Integritätsschutz umgehen. Bei diesem Beispiel wäre der Einsatz eines MACs oder von Signaturen sinnvoller.

\section{Zufallszahlen}

Bei der korrekten Verwendung von kryptographischen Methoden ist der Einsatz guter Zufallszahlengenerator essentiell. Dieser sollte Zufallszahlen mit hoher Entropie generieren. Dies kann z. B. durch Einsatz eines Hardware-Zufallsgenerators sichergestellt werden. Ist ein solcher nicht verfügbar, muss ein kryptographisch sicherer Pseudo-Zufallszahlengenerator (PRNG) verwendet werden. Moderne Betriebssysteme bieten zumeist hybride Lösungen an: hierbei werden zwar PRNGs verwendet, diese allerdings mit Entropie aus weiteren Quelle\footnote{Z. B. aus CPU-Zufallszahlengeneratoren, etc.} angereichert.

Die Qualität der generierten Zufallszahlen kann über deren Entropie bestimmt werden. Hierbei wird über statistische Methoden die Qualität der Zufälligkeit der generierten Karten ermittelt.

\section{Weitere Informationsquellen}

Entwickler benötigen Guidance zur Selektion der jeweiligen kryptographischen Algorithmen, hier eine kleine Auswahl öffentlich verfügbarer Dokumente:

\begin{enumerate}
	\item Das amerikanische NIST gibt Empfehlungen für Cryptographical Standards ab, z. B. SP-800-175B\footnote{\url{https://csrc.nist.gov/publications/detail/sp/800-175b/final}}. Aufgrund der Zusammenarbeit des NIST mit der amerikanischen NSA bei zu vorigen Crypto-Standards (Vermutung der Platzierung einer Backdoor in einen Random-Number-Generator) wird mittlerweile gerne von den NIST-Empfehlungen abgesehen.
	\item Die europäische ENISA gibt regelmäßig Empfehlungen zu verwendeten kryptographischen Standards und Schlüssellängen ab (\textit{Algorithms, key size and parameter report 2014}\footnote{\url{https://www.enisa.europa.eu/publications/algorithms-key-size-and-parameters-report-2014}}. Während diese relativ gut sind, ist die Frequenz der Veröffentlichung für IT-Verhältnisse etwas behäbig (4-5 Jahre).
	\item Das deutsche Bundesamt für Sicherheit in der Informationstechnik (BSI) bietet häufig überarbeitete Empfehlungen zum Einsatz kryptographischer Methoden an (BSI TR-02102\footnote{\url{https://www.bsi.bund.de/DE/Publikationen/TechnischeRichtlinien/tr02102/index_htm.html}}). Diese sind relativ aktuell und klassifizieren Algorithmen in sichere Algorithmen die bei aktuellen Neuentwicklungen verwendet werden sollen und in legacy-Algorithmen, die zwar nicht mehr bei Neuentwicklungen verwendet werden sollten, die aber bei bestehender Software durchaus weiterverwendet werden können.
	\item das BetterCrypto.org\footnote{\url{https://www.bettercrypto.org}} bietet regelmäßig upgedatete Beispielskonfigurationen für geläufige Webserver. Diese sollten dazu dienen, dass ein Administrator diese Snippets direkt in die Konfiguration eines Webservers kopieren können und dadurch eine sichere Konfiguration erreicht wird.
\end{enumerate}

\section{Reflektionsfragen}

\begin{enumerate}
	\item Welche Grundideen sollten bei dem Entwurf und Einsatz kryptographischer Methoden angewandt werden?
	\item Wann sollte ein MAC verwendet werden? Stelle diesen einem Hash oder einer kryptographsichen Signatur gegenüber.
\end{enumerate}
