\chapter{Sicherheit als Prozess}

Sicherheit kann nicht alleine stehen, man kann nicht ``Sicherheit programmieren'' sondern nur ``eine Applikation sicher programmieren''. Sie ist keine One-Shot Operation die einmalig vor Projektende durchgeführt wird, sondern muss während der gesamten Laufzeit der Softwareentwicklung beachtet werden. Ein typisches (fiktives) Beispiel: ein Softwareprojekt wurde als klassisches Wasserfall-Model geplant. Nach drei Jahren Laufzeit sollte das Projekt abgeschlossen sein, ca. sechs Monate vor Ende ist ein Penetration-Test der Software vorgesehen --- die sechs Monate sollten ausreichend sein um potentiell gefundene Schwachstellen auch zu beheben. Es entwickelt sich leider eine typische ``Softwareprojekt''-Geschichte: die Fertigstellung verzögert sich, schlussendlich kann der Penetration-Test erst eine Woche vor Go-Live durchgeführt werden. Natürlich werden kritische Fehler gefunden --- aufgrund der Verzögerungen und der Mehrbelastung der Entwickler war zu wenig Zeit für Sicherheits- bzw. Qualitätssicherheitsmaßnahmen vorhanden. Der Launch-Zeitpunkt kann aufgrund zugekaufter Werbung nicht mehr verschoben werden, was nun? Wie kann man diese Situation vermeiden?

Professionelle Softwareentwicklung verwendet meistens einen (semi-)standardisierten Software Development Lifecycle (SDLC), es gibt verschiedene Ausprägungen hier Security einzubringen. Zumeist werden in den jeweiligen Phasen sicherheitsrelevante Inhalte hinzugefügt:

\begin{itemize}
	\item Security Training des Personals
	\item Requirements and Risk Analysis
	\item Threat Modeling
	\item Secure Coding Guidelines, Secure Coding Checklists
	\item Security Testing Guides, Pen-Tests
	\item Vulnerability Management and Incident Response
\end{itemize}

Einige dieser Punkte werden in den Folgekapiteln etwas genauer erläutert.

\section{Requirementsanalyse}

In der \textit{Requirementsanalyse} sollte bereits Security berücksichtigt werden. Dies wird meistens unterlassen, da Security-Anforderungen non-functional\footnote{Functional Requirements beschreiben die Funktionsweise einer Applikation und sind z.B. mittels use-cases abgebildet. Non-Functional Requirements beschreiben eher die Qualität der erstellen Applikation wie Sicherheit und Performance.} requirements sind. Negative Auswirkungen dieses Versäumnis sind fehlende Awareness für Security, nicht ausreichende Ressourcen (Zeit, Personal, Budget) und schlussendlich fehlende Sicherheit im resultierenden Softwareprodukt.

\subsection{Schützenswertes Gut}
\index{Schützenswertes Gut}

Eine zentrale Frage einer Sicherheitsdiskussion ist, was überhaupt beschützt werden sollte. Diese Operationen oder Daten werden häufig \textit{Schützenswertes Gut} genannt. Beispiele für diese sind z.B. sensible Benutzerdaten, ein essentieller Geschäftsprozess aber auch immaterielle Werte wie die Reputation eines Unternehmens, dessen Aktienkurs oder intellectual property.

Häufig wird die sog. CIA-Triade\index{CIA-Triade} zur Klassifizierung verwendet. Hierbei stehen die einzelnen Buchstaben für einen schützenswerten Bereich: \textit{Confidentiality}, \textit{Integrity} und \textit{Availability}. Diese werden in Tabelle \ref{tbl:cia_triad} genauer erläutert. Die jeweiligen Bereiche sind verwandt, Availability kann stark von der Integrität der Daten abhängig sein. Beispiel: wenn eine Fahrzeitauskunft zwar als Webservice verfügbar ist, aber den Daten nicht vertraut werden kann, ist das Gesamtservice aus Usersicht wahrscheinlich nicht available.

\begin{table}
	\begin{center}
\begin{tabular}{llp{5cm}}
	\toprule
	Buchstabe & Name & Beschreibung \\
	\midrule
	C & Confidentiality & no unauthorized access to data \\
	I & Integrity & no unauthorized or undetected\footnote{Mittels Hashes, MACs, Verschlüsselung, etc. können prinzipiell Veränderungen nicht verhindert werden, allerdings können Veränderungen im Nachhinein detektiert werden.} modification \\
	A & Availability & Verfügbarkeit der Daten \\
	\bottomrule
\end{tabular}
\end{center}
\caption{CIA-Triade}
\label{tbl:cia_triad}
\end{table}

Bei realen Projekten ist die Einschätzung immer vom Kunden abhängig. Ein IT-System ist immer in die Kundenlandschaft integriert und daher können klassische IT-Fehler unterschiedliche Auswirkungen besitzen. Z. B. wird einem reinen Online-Shop die Availability wichtiger sein, als einem physikalischen Shop der nebenbei einen kleinen Onlineshop betreibt; teilweise werden Fehler durch organisatorische Maßnahme (Buchhaltung) abgefangen, etc.


\subsection{Sicheres Design}

Bei der Erstellung der Software Architektur/des Software Designs sollte auf Sicherheit geachtet werden. Um die Ziele der CIA-Triad zu erfüllen, empfehlt OWASP folgende Elemente bei der Analyse eines sicheren Designs zu beachten:

\begin{itemize}
	\item Authentication
	\item Authorization
	\item Data Confidentiality and Integrity
	\item Availability
	\item Auditing and Non-Repudiation
\end{itemize}

Die ersten vier Punkte stellen die Anforderungen aus der CIA Triade dar.

Audit Logs dienen u.a. dazu, um im Fehlerfall die Schwachstelle zu erkennen als auch den Schadfall einzugrenzen (z.B. welche User sind in welchem Umfang betroffen?). Unter Non-Repudiation versteht man die Nicht-Abstreitbarkeit: falls eine Operation von einem Benutzer durchgeführt wurde, sollte nachträglich auch verifizierbar sein, dass diese Operation auch wirklich von dem jeweiligen Benutzer in Auftrag gegeben wurde.

\section{Threat Modeling}
\index{Threat Modeling}
\label{threat_model}

Threat Models dienen zur systematischen Analyse von Softwareprodukten auf Risiken, Schwachstellen und Gegenmaßnahmen. Durch die Verwendung eines formalisierten Ablaufs wird die gleich bleibende Qualität der Analyse gewährleistet.

Bei der Analyse sollten vier Hauptfragen gestellt und beantwortet werden\footnote{Quelle: Adam Shostack --- Threat Modeling}:

\begin{enumerate}
	\item What are you building?
	\item What could go wrong?
	\item What should you do about those things that could go wrong?
	\item Did you do a decent job of analysis?
\end{enumerate}

Bevor auf diese einzelnen Bereiche kurz eingegangen wird sollte noch kurz erwähnt werden, dass Threat Models im Laufe der Zeit sehr umfangreich und daher schwer zu verstehen werden. Im Worst-Case wird es so ``aufgebauscht'' dass es nicht mehr effektiv verwendbar ist und schlussendlich nur noch ``tote'' Dokumentation darstellt. Ein guter Mittelweg zwischen Detailiertheit und Lesbarkeit ist essentiell für ein verwendbares Threat Model.

\subsection{What are you building?}

Folgende Bereiche sollten durch das Threat Model abgedeckt werden:

\begin{itemize}
	\item Threat Actors: wer sind die potentiellen Angreifer. Dies ist wichtig zu wissen, da dadurch eine bessere Ressourceneinschätzung (wie viel Zeit bzw. finanzielle Ressourcen kann ein Angreifer aufbringen?) möglich ist. Ebenso wird dadurch geklärt, ob auch Insider-Angriffe möglich sind.
	\item Schützenswertes Gut: vor welchen Angriffen hat ein Unternehmen Angst bzw. welche Daten sind schützenswert. Die Dokumentation schützenswerter Güter ergibt Synergie-Effekte zu der notwendigen DSGVO-Dokumentation.
	\item Grundlegende Sicherheitsannahmen: im Laufe eines Softwareprojektes werden Produktentscheidungen aufgrund des aktuellen Wissensstand getroffen. Hier sollten diese Entscheidungen dokumentiert\footnote{Bonuspunkte wenn nicht nur die Annahme, sondern zusätzlich auch die Auswirkungen im Falle einer gebrochenen Annahme, wer für die Überprüfung der Annahme zuständig ist, und wer fachlich die Annahme überprüfen kann, dokumentiert ist.} werden. Beispielsweise könnte für embedded systems eine schwächere Verschlüsselungstechnik gewählt worden sein, da die vorhandene Hardware nicht potent genug für ein besseres Verfahren war. Durch die Dokumentation der Annahmen können diese periodische auf ihre Haltbarkeit hin überprüft werden. Die Dokumentation dieser Annahmen ist auch essentiell im Falle des Ausfalls eines Entwicklungsteams.
	\item Scope: welche Bereiche unterliegen der Sicherheitsobacht des Entwicklers? Ist die Datenbank, der Webserver, etc. Teil des Projekts oder werden diese von externen Personen bereitgestellt?
	\item Komponenten und Datenflüsse: die Applikation wird in einzelne Komponenten dekonstruiert. Der Datenfluss (samt Klassifizierung der betroffenen Daten) zwischen den Komponenten wird meistens mittels Datenflussdiagrammen (data flow diagrams, DFDs) dargestellt.
\end{itemize}

\subsection{What could go wrong?}

Basierend auf den Datenflussdiagrammen werden potentielle Risiken und Schwachstellen identifiziert. Häufig wird hierfür STRIDE verwendet. Jeder Buchstabe dieser Abkürzung steht für eine Angriffsart, durch das Analysieren jedes Elements (des Datenflussdiagrammes) sollten möglichst viele Gefährdungen identifiziert werden. Die Tabelle \ref{tbl:stride} listet die jeweiligen Angriffsarten auf.

\begin{table}
	\begin{center}
\begin{tabular}{lp{5cm}}
	\toprule
	Buchstabe & Name\\ \midrule
	S & Spoofing \\
	T & Tampering \\
	R & Repudiation \\
	I & Information Disclosure \\
	D & Denial of Service \\
	E & Elevation of Privilege \\
	\bottomrule
\end{tabular}
	\caption{STRIDE Angriffsvektoren}
	\label{tbl:stride}
\end{center}
\end{table}

Im Privacy Umfeld existiert mit LINDDUN eine ähnliche Methode, die jeweiligen Angriffe zielen hier nun nicht auf die Sicherheit, sondern auf die Privatsphäre der Benutzer ab. Die Tabelle \ref{tbl:linddun} listet die jeweiligen Gefährdungen für die Privatsphäre auf.

Teilweise sind diese Methoden widersprüchlich. So wird im Zuge von STRIDE auf die Repudiation hin geachtet, also auf die Nicht-Abstreitbarkeit der Durchführung einer Operation, während LINDDUN dies als Non-Repudation als negativ für die Privatsphäre des Benutzers betrachtet wird.

\begin{table}
	\begin{center}
\begin{tabular}{lp{5cm}}
	\toprule
	Buchstabe & Name\\
	\midrule
	L & Linkability \\
	I & Identifiability \\
	N & Non-Repudiation \\
	D & Detectability \\
	D & Disclosure of Information \\
	U & Content Unawareness \\
	N & Policy and Consent Noncompliance \\
	\bottomrule
\end{tabular}
	\caption{LINDDUN Kategorien}
	\label{tbl:linddun}
\end{center}
\end{table}

\subsection{What should you do about those things that could go wrong?}

Die identifizierten Gefährdungen können dann mittels DREAD quantifiziert und sortiert. Diese Reihenfolge kann bei der Behebung der identifizierten Gefährdungen durch das Entwicklungsteam berücksichtigt werden.

Prinzipiell gibt es mehrere Möglichkeiten mit einer Schwachstelle umzugehen:

\begin{itemize}
	\item Elemination: die Schwachstelle wird entfernt --- dies ist effektiv nur durch Entfernen von Features möglich.
	\item Mitigation: es werden Maßnahmen implementiert die das Ausnutzen der Schwachstelle vermeiden bzw. erschweren sollen. Die meisten implementierten Sicherheitsmaßnahmen fallen in diesen Bereich.
	\item Transfer: durch Versicherungen und Verträge kann das Risiko an Andere übertragen werden.
	\item Accept: ein Risiko kann auch (durch die Geschäftsführung) akzeptiert werden. In diesem Fall ist die Dokumentation der Zuständigkeiten wichtig.
\end{itemize}

\subsection{Did we do a decent job of analysis?}

Die Ausarbeitung eines Threat Models macht Sinn wenn das Model mit der realen Applikation übereinstimmt und durch die sorgfältige Analyse der Elemente des Models Verwundbarkeiten identifiziert wurden. Die gefundenen Gefährdungen sollten in das Bug-Tracking System der Software einfließen um ein Tracking des Fortschritts zu ermöglichen.

Wird im Zuge des Softwareprojekts automatisiert getestet wird empfohlen, mittels Unit Tests die implementierten Mitigations zu verifizieren. Dadurch wird der Security Test Teil der Continues-Integration Pipeline und damit Teil der Qualitätssicherung der Software.

Zusätzlich können Penetration Tests zur Überprüfung der Sicherheit durchgeführt werden. Penetration Tests können Sicherheitsmängel aufdecken, sie sind allerdings nicht zur gezielten Erhöhung der Softwarequalität dienlich, da diese vor dem Testen bereits gewährleistet werden sollte (\textit{You can't test quality in}). Auch hier gibt es eine Interaktion mit dem Threat Model\footnote{Threatmodel: siehe Kapitel \ref{threat_model}, Seite \pageref{threat_model}}: während ein Threat Model im Gegensatz zu Penetration-Tests weniger direkte Sicherheitslücken findet, richtet es den Fokus der Penetration-Tests auf die wichtigsten bzw. gefährdetsten Komponenten der zu testenden Applikation.

\section{Secure Coding}

Während der Entwicklung sollte durch die Verwendung von Secure Code Guidelines und der Einhaltung von Best-Practises die Sicherheit der erstellten Software gewährleistet werden. Diese Maßnahmen zielen darauf ab, das Rad nicht neu zu erfinden. Durch Verwendung etablierter Methodiken und Frameworks kann auf den Erfahrungsschatz dieser zugegriffen werden und potentielle Fehler vermieden werden.

Bei der Wahl von Bibliotheken und Frameworks sollte man auf deren Security-Historie Rücksicht nehmen. Regelmäßige Bugfix-Releases mit dezidierten Security-Releases sind ein gutes Zeichen. Ebenso sind dies regelmäßige Security-Audits. Falls keine Sicherheitsinformationen verfügbar sind oder die Bibliothek/das Framework keinen langfristigen Support gewährleistet, ist dies ein Grund ggf. dieses Framework nicht zu verwenden.

Das Sicherheitslevel kann durch Verwendung von Security-Checklists überprüft werden. Ein Beispiel hierfür ist der OWASP Application Security Verfication Standard (ASVS) welcher aus einem Fragenkatalog zur Selbstbeantwortung durch Softwareentwickler besteht.

\section{Secure Testing}

Es sollte so früh wie möglich und regelmäßig wie möglich getestet werden. Zumindest vor größeren Releases sollte ein Security-Check durchgeführt werden.

Hierbei gibt es eine Interaktion mit Threat Modeling: aufgrund des Threatmodels können besonders gefährdete Bereiche identifiziert, und diese Bereiche gezielt getestet werden. Dadurch werden die Kosten des Testens reduziert.

\section{Maintenance}

Auch nach dem Abschluss der Entwicklungsphase eines Projektes gibt es Security-Anforderungen. Es sollte dokumentiert werden, wie im Falle eines Security-Vorfalls (Security Incident) vorgegangen wird. Dieser Prozess kann u.a. die Notifizierung von Kunden, das Deaktivieren von Servern, Bereitstellung eines Patch-Plans, etc. beinhalten.

Diese Vorkehrungen müssen nicht nur den eigenen Code, sondern auch Schwachstellen in verwendeten Fremdbibliotheken und Frameworks beinhalten. Angriffe gegen verwendete Bibliotheken/Frameworks (eine Form der supply-chain attacks) nahmen in letzter Zeit zu.

\section{Reflektionsfragen}

\begin{enumerate}
	\item Was versteht man unter einem Threat Model, welche Elemente sollten vorhanden sein (1-2 Sätz.B.schreibung pro Element)
	\item Welche Maßnahmen sollten im Zuge des Secure Development Lifecycles betrachtet werden? Erläutere einige der Maßnahmen.
\end{enumerate}

