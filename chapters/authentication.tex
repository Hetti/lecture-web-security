\chapter{Authentifikation}
\index{Authentication}

Sobald eine Webapplikation sensitive bzw. privilegierte Operationen und Daten bereitstellt, besteht die Notwendigkeit die Identität des zugreifenden Benutzers zu erheben und zu verifizieren.

Authentifikation kann als die Verifikation einer behaupteten Benutzeridentität über zuvor ausgetauschte Details (wie z.B. das während der Registrierung angegebenen Passwort) definiert werden. Nach erfolgtem Login wird zumeist eine langfristige Verbindung (Session) zu dem Benutzer aufgebaut. Bei Folgezugriffen wird dieses Vertrauensverhältnis verwendet, um den Benutzer sowohl zu identifizieren als auch authentifizieren.

\section{Identifikation und Authentifikation}

Bei der Identifikation claimed der Benutzer seine Identität, z.B. durch Angabe eines zuvor registrierten Benutzernamens. Weitere Möglichkeiten wären z.B. SmartCards oder biometrische Methoden. Die Identifikation wird zumeist mit einer Authentifikation kombiniert.

Die Authentifikation dient zur Validierung der behaupteten Identität des Benutzers. Es gibt mehrere Möglichkeiten (Faktoren) über welche ein Benutzer seine Identität authentifizieren kann, Tabelle \ref{tbl:factors} gibt eine kurze Übersicht häufig genutzter Faktoren.

\begin{table}[h!]
	\begin{center}
\begin{tabular}{lp{7cm}}
	\toprule
	Faktor & Art\\
	\midrule
	Passwort & Something you know \\
	Hardware-Tokens & something you have \\
	Biometrie & something you are \\
	Soziale Beziehungen & someone you know \\
	Email-Konto & z.B. Slackanmeldung mittels Link in Email \\
	PostIdent & Verifikation am Postamt mittels Ausweis \\
	VideoIdent & Verifikation mit Ausweis mittels Videokonferenz \\
	PhotoIdent & Verifikation über zugeschicktes Ausweisbild \\
	\bottomrule
\end{tabular}
	\caption{Verschiedene Faktoren zur Authentication}
	\label{tbl:factors}
\end{center}
\end{table}

Bei der initialen Registrierung und bei nachfolgenden Anmeldungen können unterschiedliche Faktoren verwendet werden. Z. B. VideoIdent bei der Registrierung, bei Folgeanmeldungen Passwörter.

Durch die Kombination mehrere Faktoren erhält man eine Multifaktor-Authentifikation (MFA), häufig wird als Zweifaktoren-Authentifikation (2FA) ein Passwort mit einem Token kombiniert. Wichtig bei der MFA ist die Wahl von Faktoren aus unterschiedlichen Klassen. Es macht z.B. wenig Sinn eine Fingerprint-Authentifikation mit einer Iris-Authentifikation zu kombinieren. Ein schönes Beispiel, bei dem Faktoren verschiedener Klassen schlecht durch einen User kombiniert werden wäre es, wenn der Benutzer einer Bankomatkarte seinen PIN (something you know) auf seine Bankomatkarte (something you have) schreibt.\index{Multi-Faktoren-Authentication}\index{Zwei-Faktoren-Authentication}\index{2FA}

\section{Login- und Logout}
\index{Login- und Logout}

Wenn ein Login- und Logout innerhalb der Applikation implementiert werden, müssen gewisse Grundfunktionen abgedeckt sein.

\subsection{Login-Formular}

Das Login-Formular sollte entsprechend dem KISS-Prinzip als einfaches HTML-Formular implementiert werden. Hauptgrund dafür ist, dass das Login-Formular mit Passwort-Managern kompatibel sein sollte. Dies impliziert, dass das Login-Formular aus Textfeldern für Benutzername und Passwort als auch einem Login-Button bestehen sollte.

Negative Beispiele die den Einsatz von Passwort-Managern erschweren:

\begin{itemize}
	\item Benutzername und Passwort-Feld sind nicht innerhalb der gleichen Seite
	\item Password-Feld wird erst angezeigt, nachdem ein Benutzername eingegeben wurde
	\item Verwendung von Flash-, Silverlight- oder Java-Applets
	\item Authentication through EMail a la Slack (Email mit Bestätigungslink dient als Passwortersatz)
	\item HTTP BASIC basierte Authentifikation
\end{itemize}

\subsection{User Enumeration Angriffe}
\index{Login- und Logout!User Enumeration}
\index{User Enumeration}

Eine User Enumeration liegt vor, wenn ein Angreifer gezielt Informationen über das Vorhandensein eines Benutzers erzielen kann. Zumeist geschieht dies über schlecht gewählte Fehlermeldungen. So kann ein Angreifer bei der Fehlermeldung ``Passwort invalid'' davon ausgehen, dass ein Benutzername dem System bekannt ist. Lösung: Verwendung generischer Fehlermeldungen wie ``Benutzer/Passwort-Kombination nicht bekannt''.

Während dies bei einem Login-Formular leicht zu bewerkstelligen ist, sind weitere Operationen komplexer:

\begin{itemize}
	\item ``Passwort vergessen''-Funktion: hier muss meistens eine Email-Adresse angegeben werden. Falls die Email-Adresse dem System nicht bekannt ist, sollte keine Fehlermeldung ausgegeben werden, sondern ein Hinweis, dass an die angegebene Email eine Benachrichtigungsemail versendet wurde.
	\item Bereits vorhandene Email-Adresse bei Registrierung: hier sollte ebenso eine neutrale Erfolgsmeldung innerhalb der Webseite ausgegeben, und anschließend in einer Bestätigungsemail der Benutzer darauf hingewiesen werden, dass er bereits ein Konto mit der Email-Adresse angelegt hatte.
	\item Bereits vorhandener Login bei Registrierung: hier muss dem User eine Fehlermeldung angezeigt werden.
\end{itemize}

Generell ist dieser Bereich einer derjenigen, bei denen Usability und Security potentiell konträre Ziele besitzen.

\subsection{Brute-Force Angriffe gegen Login-Formular}
\index{Login- und Logout!Brute Force Angriffe}
\index{Brute Force Angriffe}

Brute-Force Angriffe versuchen mittels automatisierter Anfragen eine valide Kombination von Benutzernamen und Passwort zu erraten. Durch die Kenntnis bekannter Benutzernamen können Brute-Force Angriffe beschleunigt werden (z.B. durch eine zu vorige User Enumeration).

Technisch sind Brute-Force Angriffe einfach umzusetzen, Tool-Support ist massiv vorhanden. Die erreichte Geschwindigkeit befindet sich meistens bei mehreren Zehntausend Versuchen pro Stunde.

Brute-Force Angriffe versuchen entweder eine Kombination des gesamten Testbereichs (Buchstaben, Zahlen, Sonderzeichen) oder verwenden vorbereitete Passwortlisten. Diese können auf verschiedene Arten bereitgestellt werden:

\begin{itemize}
	\item Sammlung von Passwörtern von etwaigen Password Leaks.
	\item Automatisch generierte Liste basierend auf den öffentlichen Seiten der zu testenden Homepage.
	\item Deep-Learning basierte Verfahren, die basierend auf existierenden Passwortlisten neue Passwortlisten generieren.
\end{itemize}

Gegenmaßnahmen zielen auf eine Verlangsamung des Angriffs bzw. auf eine Sperre betroffener Konten ab:

\begin{itemize}
	\item Rate-Limits bzw. Verlangsamung bei Fehlerseiten.
	\item Sperre von Benutzeraccounts bzw. IP-Adressen nach einer definierten Anzahl von Fehlversuchen.
	\item Einsatz einer Mehrfaktorauthentication. Durch die benötigte manuelle Interaktion wird eine Brute-Force Attacke ausgebremst. Hier ist die Wahl eines geeigneten Faktors und eine geeignete Integration notwendig.
\end{itemize}

\subsection{Logout}

Symmetrisch zum Login sollte auch eine Logout-Operation implementiert werden. Dadurch kann der Benutzer seine Session beenden und dadurch das mögliche verwendbare Zeitfenster gegenüber Angriffen (z.B. gegenüber CSRF-Angriffen) verkleinern.

Bei neueren Standards wie der ÖNORM A77.00 gibt es die Anforderung, dass der Benutzer nicht nur seine aktuelle, sondern auch alle seine bestehenden Sitzungen beenden kann.

Beispiel: Benutzer besitzt einen Desktop und einen Laptop. Der Laptop wird gestohlen, es sollte möglich sein eine offene Web-session am Laptop über den Desktop zu beenden.

\subsection{Deaktivieren/Sperren/Löschen von Accounts}
\index{Login- und Logout!Sperren von Accounts}

Wird ein Benutzeraccount gelöscht oder deaktiviert stellt sich die Frage, wie mit den gelöschten Daten des Benutzers umzugehen ist. Wurde ein Account gesperrt muss dafür Sorge getragen werden, dass:

\begin{itemize}
	\item bereits ausgestellte Recovery-Codes den Account nicht reaktivieren können
	\item aktive Benutzersessions beendet werden
	\item der Benutzer sich nicht mehr einloggen kann
\end{itemize}

Die Hauptfrage bei einem zu löschenden Account ist, welche Daten gelöscht, und welche Daten persistiert werden müssen (beides primär aus rechtlichen Gründen).

\section{Behandlung von Passwörtern}
\index{Passwörter}

Die grundsätzliche Strategie wäre, keine Passwörter in der Applikation zu speichern, einzugeben oder zu verarbeiten. Wenn die Applikation niemals Zugriff auf Passwörter hat, können diese auch nicht verloren werden. Falls dies nicht möglich ist, müssen beim Umgang mit Passwörtern gewisse Grundregeln eingehalten werden.

Genauere Informationen zur sicheren Speicherung von Passwörtern können im Kapitel \textit{Sensitive Data Exposure} (\ref{password_storage}) gefunden werden.

Prinzipiell können Angriffe gegen Passwörter in drei Kategorien eingeteilt werden:

\begin{enumerate}
	\item Disclosure tritt auf, wenn das Passwort unbeabsichtigt ``veröffentlicht'' wird. Dies kann z.B. durch Notizzettel, Wikis oder auch durch phishing geschehen.
	\item Online Attacks sind Angriffe gegenüber einem Login-System. Diese können durch das Websystem erkannt werden.
	\item Offline Attacks sind Angriffe gegenüber geleakten Passwort-Hashes. Diese können durch das Websystem nicht erkannt werden.
\end{enumerate}

\subsection{Passwort-Qualität}
\index{Passwörter!Qualität}

Kann ein neues Passwort in der Applikation gesetzt werden, sollte dieses gewisse Mindestanforderungen erfüllen. 2018 wurden die NIST 800-63-3: Digital Identity Guidelines\footnote{\url{https://pages.nist.gov/800-63-3/}} veröffentlicht, diese inkludieren mehrere Best-Pracises im Umgang mit Passwörtern:

\begin{itemize}
	\item Minimale Passwortlänge: 8 Zeichen. Ein Unicode Zeichen ist ein Zeichen.
	\item Falls ein Benutzer ein längeres Passwort eingibt, müssen mindestens 64 Zeichen gespeichert werden.
	\item Das periodische Neusetzen von Passwörtern wird nicht mehr gefordert. Diese Maßnahme bewirkte schwächere Passwörter.
	\item Komplexitätsregeln bei Passwörtern (mindestens ein Sonderzeichen und ähnliches) wurden entfernt.
	\item Neu eingegebene Passwörter müssen gegen eine Liste von bekannten Passwort-Leaks und gegen bekannte Standard bzw. häufig genutzte Passwörter getestet werden.
	\item Passwort-Hints dürfen nicht mehr verwendet werden.
\end{itemize}

Um eingegebene Passwörter gegen eine Liste von geleakten Passwörtern zu überprüfen, kann z.B. von \url{https://haveibeenpwned.com} (im Folgenden immer haveibeenpwned genannt) eine ca. 10 Gigabyte große Liste an Passwort-Hashes heruntergeladen werden. Alternativ bietet haveibeenpwned einen Passwort-Check Service an. Bei diesem werden Passwörter nicht als Hash übermittelt (ansonsten würde der Serverbetreiber Wissen über die verwendeten Passwörter erhalten), sondern es wird das Passwort gehashed, die ersten 5 Zeichen des Hashes übertragen und anschließend eine Liste aller gefundenen Hashes an den Client zurück übertragen.

\subsection{Passwort-Reset}
\index{Passwörter!Vergessen Funktion}

Ein wichtiger Grundsatz ist \textit{Account recovery not password recovery}. Dieser sagt aus, dass der Benutzer wieder Zugang zu seinem Account erhält, aber nicht sein bestehendes Passwort einsehen kann. In einer korrekt implementierten Applikation sollte das bestehende Passwort nirgends unverschlüsselt gespeichert werden, daher sollte diese Möglichkeit prinzipiell nicht technisch möglich sein.

Meistens wird man aus Gründen der Usability dem User eine Möglichkeit des Passwort-Resets geben. Dies wird normalerweise über eine Email mit einem Passwort-Reset Link implementiert. Folgende Implementierungshinweise:

\begin{itemize}
	\item Dem User sein bestehendes Passwort zuzusenden ist ein epic fail da hierfür das Passwort unverschlüsselt gespeichert werden müsste.
	\item Dem Benutzer ein neues Passwort per Email zuzuschicken sollte vermieden werden.
	\item Der generierte Link sollte nur einmalig verwendbar sein, und auch nur das Updaten des aktuellen (vergessenen) Passworts erlauben.
	\item der generierte Link sollte nur für den betreffenden User verwendbar sein.
\end{itemize}

Hinweis: die aktuellen NIST Richtlinien verbieten explizit die Verwendung von ``Passwort Fragen'' (``In welcher Straße bist du aufgewachsen, etc.'') zum Zurücksetzen des Passworts. Grund: diese Fragen waren bei bekannteren Personen einfach nachzuforschen.

Die Verifikation kann auch über Alternate Transports geschehen. Ein Beispiel wäre die österreichische Sozialversicherung, bei der ein neues Passwort über einen eingeschriebenen Brief an den User verschickt wird. Dadurch wird eine Identitätsfeststellung des Empfängers erzwungen.

Sobald ein neues Passwort gesetzt wurde sollte der Benutzer über mehrere Wege über diese Passwortänderung notifiziert, und ihm eine Möglichkeit der Account-Sperre gegeben, werden. Typische Nachrichtenwege wären z.B. Emails oder SMS.

\subsection{Ändern von Passwörtern}

Der Benutzer sollte die Möglichkeit besitzen, sein Passwort neu zu setzen. Für eine sichere Operation muss folgendes gegeben sein:

\begin{itemize}
	\item der User muss aktuell authenticated sein
	\item der Benutzer kann nur sein eigenes Passwort ändern
	\item im Zuge der Operation, die das neue Passwort setzt, muss auch das alte Passwort erfragt werden.
\end{itemize}

Das bestehende Password wird erfragt, damit ein Angreifer mit Zugriff auf die Session nicht ein neues Passwort setzen kann (und dadurch unbegrenzten Zugriff auf das Benutzerkonto erhält). Im einfachsten Fall geschieht so ein Angriff indem der Angreifer auf einem nicht-gesperrten Computer ein neues Passwort innerhalb einer eingeloggten Webapplikation eingibt.

Damit diese Schutzmaßnahme funktioniert, müssen sowohl das alte als auch das neue Passwort im gleichen Schritt übermittelt werden. Ebenso verhindert dies CSRF-basierte Angriffe.

\subsection{Passwörter für Dritt-Dienste}

Teilweise können Passwörter nicht gehashed innerhalb der Applikation gespeichert werden. Dies tritt zum Beispiel auf, wenn das Passwort an eine Drittapplikationen weitergegeben werden muss -- eine Webapplikation welche zur Darstellung eines IMAP-Emailkontos dient muss z.B. innerhalb der Applikation die Zugangsdaten für das externe Email-System speichern. Falls dieser Email-Server das Passwort in plain-text benötigt, muss die Applikation nun auch das Passwort in plain-text speichern und kann daher keine Einweg-Hashfunktion anwenden.

Prinzipiell ist hier das Grundproblem, dass das sensible Passwort an eine externe, potentiell nicht vertrauenswürdige, Applikation übergeben werden muss.

\section{Alternativen zu Passwort-basierten Logins}

Benutzer sind notorisch schlecht bei der Wahl sicherer Passwörter. Um diese Gefahrenquelle zu minimieren wird versucht, entweder die Sicherheit des Login-Vorgangs mit einem zweiten Faktor zu verstärken, oder Passwörter vollkommen durch physikalische Tokens zu ersetzen.

TOTP ist ein Verfahren, dass zur Implementierung eines zweiten Faktors eingesetzt werden kann. Im Gegensatz dazu, werden die Protokolle der FIDO-Allianz häufig für die Implementierung Passwort-loser Authentifizierungsvorgänge genutzt.

\subsection{TOTP}
\index{TOTP}

Time-based One-Time Passwords (TOTP, RFC 6238) ist ein häufig verwendetes Verfahren zur Implementierung eines weiteren Authentication-Faktors. Es ist ein Zusammenspiel zwischen Authenticator (meist eine mobile Applikation) und einer Webapplikation.

Initial wird ein shared secret key zwischen Authenticator und Webapplikation ausgetauscht. Wird nun eine Authentifikation benötigt wird nun auf beiden Seiten die aktuelle Systemzeit (in Sekunden seit Beginn der UNIX Epoche) auf 30 Sekunden gerundet und ein MAC (unter Zuhilfenahme des shared secret keys) gebildet. Dieser MAC wird nun auf 31 bit gekürzt und in einen 6 oder 8 stelligen Zahlencode verwandelt. Dieser wird am Authenticator angezeigt und muss vom User in der Webapplikation als weiterer Faktor eingegeben werden. Sofern beide berechnete Werte ident sind, wird die Authentifikation erfolgreich durchgeführt.

Ein Vorteil dieses Verfahrens ist, dass nach dem initialen Schlüsselaustausch keine Netzwerkverbindung zwischen Authenticator und Applikation benötigt wird. Ein Nachteil ist, dass die Systemuhren der betroffenen Systeme synchronisiert werden müssen. Ebenso kann bei TOTP kein \textit{device-binding} durchgeführt werden: die Webapplikation kann nicht feststellen, auf wie vielen devices ein TOTP-Secret eingespielt wurde. Ebenso ist der Vorgang der initialen Secret-Verteilung gefährlich: wird hier z.B. von einem Benutzer ein Selfie inklusive dem QR-Code/Secret-Code erstellt und veröffentlicht, wurde auf diese Weise die gesamte Sicherheit des Verfahrens kompromittiert.

\subsection{Protokolle der FIDO-Alliance}
\index{FIDO}

Die FIDO-Alliance ist eine nicht-kommerzielle Vereinigung von über 150 Unternehmen und Behörden mit dem Ziel, offene und lizenzfreie Authentifizierung-Industriestandards zu schaffen. Ihre Mitglieder beinhalten u.a. Alibaba, Google, Microsoft, Samsung und YubiCo. Die Abkürzung FIDO steht dabei für \textit{Fast IDentity Online}.

Ende 2014 wurde FIDO 1.0 veröffentlicht, dieser Standard umfasste:

\begin{itemize}
	\item U2F (Universal Second Factor) standardisiert den Einsatz von physikalischen Tokens (wie z.B. einem Yubikey). Sofern die Webapplikation und der verwendete Webbrowser U2F unterstützen kann der Benutzer sich mit einem Hardware-Token authentifizieren (z.B. durch Knopfdruck auf einem USB-Stick oder durch Antappen eines NFC/BLE Tokens).
	\item UAF (Universal Authentication Framework) dient zur Implementierung eines Passwort-losen Logins. Der Benutzer muss über ein UAF-kompatibles Endgerät verfügen (z.B. Windows 10) und registriert quasi sein Endgerät bei der Webapplikation.
\end{itemize}

Das Grundprinzip basiert auf public key Kryptographie. Wenn ein Authenticator (z.B. Android Gerät) als Gerät eines Benutzers registriert wird, wird im Gerät ein public/private key pair generiert und der public key dem FIDO Server mitgeteilt. Im Falle einer Benutzerauthentication wird die Anfrage des Servers vom Client mit dem private key signiert, mit dem serverseitig hinterlegten public key verglichen und damit die Identität des Benutzers verifiziert. Lokal werden meistens biometrische Methoden zum Schutz der Tokens verwendet.

FIDO2 kombiniert mehrere Projekte um eine passwortlose Authentication zu erlauben. Das vom W3C standardisierte WebAuthn wird von Webbrowsern implementiert und erlaubt es Webapplikationen (mittels JavaScript) eine FIDO Benutzerauthentication durchzuführen. Das Client-to-Authenticator-Protocol (CTAP) standardisiert das Kommunikationprotokoll zwischen Authenticator (Hardware-Tokens) und dem Webbrowser (Client). Es gibt zwei Varianten CTAP1 und CTAP2 wobei CTAP1 dem FIDO U2F Standard entspricht.

\subsection{Gegenüberstellung FIDO/TOTP}

Wird FIDO mit TOTP verglichen, können konzeptionelle Unterschiede erkannt werden:

\begin{itemize}
	\item FIDO1/2 überträgt nur einen öffentlichen Schlüssel während der Registrierung eines neuen Authenticators. TOTP überträgt ein shared secret. Bei FIDO verlässt der geheime Schlüssel niemals den Authenticator.
	\item TOTP benötigt im Gegensatz zu FIDO während der Authentifizierung keine aktive Netzwerkverbindung zwischen Authenticator und Service. Stattdessen benötigt TOTP eine synchronisierte Systemzeit zwischen allen beteiligten Parteien.
	\item Da bei FIDO der geheime Schlüssel nicht den Authenticator verlässt, gibt es ein Pairing zwischen dem Device und dem Service. Bei TOTP kann ein Benutzer das idente shared secret mit mehreren Authenticators verwenden, eine Zuordnung zu einem dedizierten Authenticator ist daher nicht möglich.
	\item TOTP besitzt keine Hardware-Requirements und kann daher gratis in Software implementiert werden. Während FIDO ein freier Standard ist, setzt es einen Hardware-Token voraus --- dadurch ist der Einsatz von FIDO mit Hardware-Kosten verbunden und ist tendenziell nicht ``gratis''.
\end{itemize}

\section{Authentication von Operationen}

Um eine serverseitige Rechtekontrolle durchführen zu können ist sowohl eine Benutzer-Identifikation, -Authentifikation und Authorization notwendig. Es muss sowohl die Benutzeridentität als auch dessen Berechtigung (Authorization) überprüft werden. Dies muss vor Exekution der aufgerufenen Operation serverseitig durchgeführt werden.

Da für jede Überprüfung der Authorization eine Feststellung der Benutzeridentität notwendig ist, werden beide meistens zusammengefasst durchgeführt. Ein wichtiger Unterschied ist, dass bei einem Fehler innerhalb der Authorization ein Angreifer eine Operation trotz fehlender Berechtigung aufrufen kann, dieser Aufruf allerdings einem bestehenden Benutzerkonto zugeordnet werden kann (audit/log trail). Bei Fehlern in der Authentifikation kann jeder anonyme Internetbenutzer auf die betroffenen Operationen und Daten zugreifen. Dies gilt auch für automatisierte Scantools, Search Bots und Crawler. Falls bei der Transportlevel-Sicherheit auch Fehler vorhanden sind, besteht ebenso großes Risiko durch Man-in-the-Middle Proxies (MitM-Proxies).

\subsection{Probleme im Umfeld der Authentication}

Das schwerwiegendste Problem wäre es, wenn keine Kontrolle der Authentication durchgeführt wird. Nach einem Login kann jeder anonyme Benutzer auf alle Operationen und Daten zugreifen, eine Zuordnung der Operationsausführung zu einem eingeloggten Benutzer findet nicht statt. Prinzipiell handelt es sich hierbei um Security-by-Obscurity da die Sicherheit der Operationen und Daten nur davon abhängt, dass der Angreifer die URL der Operationen nicht kennt. Dies ist allerdings selten der Fall, da MitM-Proxies und Crawler zur Identifikation der Operationen verwendet werden können. Ebenso stellen die meisten API-Server automatisch generierte Dokumentation der bereitgestellten Operationen zur Verfügung.

In abgeschwächter Form kann eine ähnliche Schwachstelle teilweise bei historisch gewachsenen Applikationen gefunden werden. Hier haben sich im Laufe der Zeit Technologietrends, Firmen-Guidelines oder Programmierteams verändert und die Gesamtapplikation besteht aus Komponenten, die in verschiedenen Programmiersprachen/Frameworks implementiert wurden. Da Authentication-Daten zumeist über das Framework abgehandelt werden, passiert es hier nun häufig, dass bei einem Teil der Applikationen die Authentication vergessen wird.

Ein weiteres häufiges Problem sind selbst geschriebene Komponenten. Ähnlich wie bei dem letzten Fehlerfall wird hier eine bestehende Applikation um eine weitere Funktion erweitert, auch hier kann dies zeit verzögert durch neue Programmierer geschehen. Bei den neu geschriebenen Komponenten wird gerne auf die Authentication vergessen --- eine mögliche Ausrede wäre es, dass externe Programmierer eventuell das bestehende System nicht gut kennen.

Beispiel: eine Kundenwebseite erlaubt den Download von Rechnungen. Die gesamte Webseite ist in JSP geschrieben, die Downloadseite allerdings in ASP.Net. Rechnungen können über die URL /documents/download/123 bezogen werden. Bei Tests wurde festgestellt, dass über freie Wahl der ID (Zahl) beliebige Kundenrechnungen heruntergeladen werden konnten, da keine Authentication implementiert wurde. Bei Analyse der Logdateien wurde weiters festgestellt, dass die betroffenen Daten bereits vom Google SearchBot indiziert wurden und somit im Suchindex aufgenommen waren.

Gegenbeispiel: die Webseite eines Personentransportunternehmens verschickt eine Email mit einen Link auf das gekaufte Ticket. Beim Ticket-Download wird keine Authentication durchgeführt, Begründung: bei der Kontrolle gab es immer wieder Probleme, dass Kunden ihr Ticket nicht herunterladen konnten da sie ihre Zugangsdaten vergessen hatten. Um das Risiko zu senken wurden als IDs große Zufallszahlen gewählt.

\section{Reflektionsfragen}

\begin{enumerate}
	\item Was versteht man unter Multi-Faktor-Authentication?
	\item Wie funktionieren TOTP und FIDO U2F? Worin liegen konzeptionelle Unterschiede?
	\item Welche Regeln sollten bei der Speicherung von Passwörtern und zu der Sicherung der Qualität der Passwörter beachtet werden?
	\item Was ist der Unterschied zwischen Identification und Authentication? Nenne zumindest vier Beispiele wie ein Benutzer identifiziert werden kann.
	\item Wie sollte ein Login-Formular gestaltet sein? Von welchen Techniken sollte man Abstand nehmen?
	\item Was ist eine User Enumeration und wie kann man sich dagegen schützen? Was sind komplexere Applikationsfunktionen die schwer gegenüber User Enumeration absicherbar sind?
	\item Was sind Brute-Force Angriffe und wie werden die dabei verwendeten Daten erzeugt? Welche Gegenmaßnahmen gibt es?
	\item Auf welche Gefahren sollten bei der Implementierung der Passwort-Vergessen Funktion geachtet werden?
\end{enumerate}
