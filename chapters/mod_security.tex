\chapter{WAF: mod\_security}

\textit{mod\_security} ist eine Web-Application Firewall --- dies bedeutet, dass sie laut IOS/OSI Model auf Applikationsebene (Level 7) arbeitet und daher auch die Sprache der Applikation (in diesem Fall HTTP) spricht und versteht. Sie ist eine der bekanntesten Open-Source WAFs und wird daher im Zuge dieses Skripts betrachtet.

Laut dem eigenen Selbstverständnis bietet sie \textit{Real-Time Application Security Monitoring and Access Control}: ein- und ausgehende HTTP Anfragen und Antworten können in Echtzeit analysiert und aufgrund von Regeln kontrolliert (z. B. blockiert) werden. \textit{mod\_security} blockiert by Default keinen Traffic, alle Entscheidungen müssen explizit über Regeln konfiguriert werden. Ein häufig verwendeter Regelsatz ist de OWASP Core Rule Set (CRS). Die WAF führt kein Input-Sanititzing durch: ein Request wird also entweder akzeptiert oder blockiert. Dies war eine Design-Entscheidung des Entwicklers um sich auf eine Kernaufgabe zu konzentrieren und diese gut zu lösen (do one thing and do this well).

\textit{mod\_security} kann entweder als Apache Modul (daher der Name= innerhalb eines bestehenden Apache Webservers integriert werden, oder als eigenständiger Reverse-Proxy (als eigenständiger Apache-Server) vor mehrere Applikationsserver vorgeschalten werden.

Wie bereits erwähnt ist der primäre Einsatzzweck von \textit{mod\_security} ist das Monitoren und Kontrollieren von HTTP Traffic in Echtzeit und das Erstellen von Zugriffsentscheidungen aufgrund dieses Traffics. Konkreter kann z. B. \textit{Virtual Patching} durchgeführt werden. Dabei wird eine Sicherheitslücke gepatched ohne, dass die Web-Applikation selbst angepasst wird. Beispielsweise könnte es sich bei dem Sicherheitsproblem um einen 0day (ohne verfügbaren Patch), oder bei der Applikation um eine ungewartete proprietäre Applikation handeln. In diesem Fall würde das Angriffsmuster analysiert und für den Angriffsvektor eine Regel hinterlegt werden, welche bösartige Angriffe automatisiert erkennt und die jeweiligen Requests blockiert. Eine weitere Möglichkeit ist die Verwendung vordefinierter Regelwerker zum Hardening bestehender Applikationen (dadurch wird eine weitere Schutzschicht über eine eigentlich sichere Applikation eingezogen).

Eine unterschätzte Möglichkeit von \textit{mod\_security} ist der Einsatz als Log-Quelle. Webserver protokollieren im Normalfall nur die zugegriffene URL (also keinen Request-Body und auch keine Antwortdokumente). Mittels \textit{mod\_security} kann ein Administrator nun genau definieren, welche Requests in welcher Tiefe protokolliert werden sollten.

\section{Functionality and Transactions}

Grundsätzlich kann die implementierte Funktionalität in vier Bereiche aufgeteilt werden:

\begin{itemize}
	\item parsing: eingehende und ausgehene HTTP Nachrichten müssen analysiert werden.
	\item buffering: teilweise werden Nachrichten auf mehrere Teilnachrichten aufgeteilt. \textit{mod\_security} sammelt diese, fügt diese zu einer gemeinsamen Nachricht zusammen und ruft erst danach die Filterregeln auf.
	\item rule-engine: ist die aktive Komponente welche die bereitgestellten Regeln auf die gesammelten HTTP-Nachrichten anwendet.
	\item logging: führt das Logging durch.
\end{itemize}

Hier kann man auch schon den Overhead von \textit{mod\_security} erkennen: auf der einen Seite gibt es einen CPU-Overhead da alle Nachrichten analysiert werden müssen, auf der anderen Seite müssen für das Buffering alle ein- und aus-gehenden Nachrichten im Speicher gesammelt werden und führen so zu Speicher-Overhead. Falls zuviel Speicher verbraucht wird, müssen Nachrichten auf der Festplatte/SSD zwischengespeichert werden. Beides erhöht die Latenz-Zeit die ein Benutzer wahrnimmt.

Ein eingehender HTTP Request und die dazugehörende Response durchläuft innerhalb von \textit{mod\_security} folgende fünf Phasen:

\begin{enumerate}
	\item request headers : cheap, decide if you want to look at body content
	\item request body : think POST-Requests, attached Files
	\item response headers : cheap, decide if you want to look at body content
	\item response body, attached Files
	\item logging : decide if the request/response will be logged or not
\end{enumerate}

In Phase 1 und 2 kann das Weiterleiten des Requests zum Applikationsserver blockiert werden. Innerhalb der phase 3 und 4 kann das Antwortdokument zum Client blockiert werden, die Operation wird allerdings am Applikationsserver exekutiert (dies kann z. B. verwendet werden um Data-Extraktion zu verhindern). Inder Phase 5 kann ein Request nicht mehr blockiert werden.

\subsection{Beispielstransaktion}

Anhand der Debug-Logs von \textit{mod\_security} wird hier nochmal eine Transaktion erklärt.

Folgender Input Request mit jeweils einem Parameter in der URL (\textit{a}) und einem Parameter im Body (\textit{b}).

\begin{minted}{http}
POST /?a=test HTTP/1.0
Content-Type: application/x-www-form-urlencoded
Content-Length: 6

b=test
\end{minted}

Dies führt zu folgendem Antwortdokument mit der Nachricht \textit{Hello World!} im Content:

\begin{minted}{http}
HTTP/1.1 200 OK
Date: Sun, 17 Jan 2010 00:13:44 GMT
Server: Apache
Content-Length: 12
Connection: close
Content-Type: text/html

Hello World!
\end{minted}

Folgende Phasen können nun in den Log-Dateien erkannt werden:

\begin{minted}{text}
[4] Initialising transaction (txid SopXW38EAAE9YbLQ).
[5] Adding request argument (QUERY_STRING): name "a", value "test"
[4] Transaction context created (dcfg 8121800).
[4] Starting phase REQUEST_HEADERS.
\end{minted}

Für jeden eingehenden HTTP-Request wird zunächt eine eindeutige Transaktions-ID vergeben. Anschließend wird der HTTP Header analysiert und der Parameter \textit{a} extrahiert. Abschließend werden die Regeln der Phase 1 (Request-Header) angewandt.

\begin{minted}{text}
[4] Second phase starting (dcfg 8121800).
[4] Input filter: Reading request body.
[9] Input filter: Bucket type HEAP contains 6 bytes.
[9] Input filter: Bucket type EOS contains 0 bytes.
[5] Adding request argument (BODY): name "b", value "test"
[4] Input filter: Completed receiving request body (length 6).
[4] Starting phase REQUEST_BODY.
\end{minted}

Falls die Nachricht noch nicht blockiert wurde, wird nun der Content des Requests analysiert und dabei der Parameter \textit{b} extrahiert. Jetzt besitzt \textit{mod\_security} Zugriff auf alle Parameter und die Regeln der Phase 2 (Request Body) werden angewendet.

\begin{minted}{text}
[4] Hook insert_filter: Adding input forwarding filter (r 81d0588).
[4] Hook insert_filter: Adding output filter (r 81d0588).

[4] Input filter: Forwarding input: mode=0, block=0, nbytes=8192 (f 81d2228, r 81d0588).
[4] Input filter: Forwarded 6 bytes.
[4] Input filter: Sent EOS.
[4] Input filter: Input forwarding complete.
\end{minted}

Wenn auch diese Regeln den Request nicht verwerfen, wird der Request an den Applikationsserver/die Applikation weitergeleitet. \textit{mod\_security} kann nun abwarten, bis ein entsprechendes Antwortdokument empfangen wird.

\begin{minted}{text}
[9] Output filter: Receiving output (f 81d2258, r 81d0588).
[4] Starting phase RESPONSE_HEADERS.

[9] Output filter: Bucket type MMAP contains 12 bytes.
[9] Output filter: Bucket type EOS contains 0 bytes.
[4] Output filter: Completed receiving response body (buffered full - 12 bytes).
[4] Starting phase RESPONSE_BODY.
\end{minted}

Analog zu der Request-Phase wird nun das Antwortdokument analysiert und die Regeln der jeweiligen Phasen (3 und 4) angewendet. Hier sieht man auch schön, dass auch das Antwortdokument gespeichert wird --- werden z. B. viele große Dokumente zugestellt (oder Dateien heruntergeladen) müssen diese zwischengespeichert, und auf diese Weise serverseitig viele Ressourcen verbraucht, werden.

\begin{minted}{text}
[4] Output filter: Output forwarding complete.
[4] Initialising logging.
[4] Starting phase LOGGING.
[4] Audit log: Ignoring a non-relevant request.
\end{minted}

Abschließend wird noch die Log-Phase durchspielt und aufgrund dieser erkannt, ob der Request bzw. die Response in die Logdatei aufgenommen werden sollte.

\section{Rules}

Das Herz einer \textit{mod\_security} Installation sind die verwendeten Filterregeln. Diese besitzen immer das idente Format:

\begin{minted}{text}
SecRule ARGS OPERATOR ACTIONS
\end{minted}

Die jeweiligen Bereiche sind einfacher erklärt:

\begin{description}
	\item[ARGS]: beschreibt auf welchen Bereich (z. B. auf welchen Parameter) eine jeweilige Operation angewendet werden soll.
	\item[OPERATOR]: definiert die Abfrage/den Filter der auf die ARGS angewandt werden soll.
	\item[ACTIONS]: beschreibt die Aktivitäten (z. B. blockieren und loggen) die angewandt werden, falls der Operator erfolgreich auf die Argumente angewandt werden konnte.
\end{description}

\subsection{Rules: Args}

Beispiele für mögliche Argumente:

\begin{itemize}
	\item ARGS (GET+POST arguments)
	\item ARGS\_GET, ARGS\_POST: die Parameter dre Header/des Contents
	\item FILES: hochgeladene Dateien
	\item FULL\_REQUEST: der gesamte Request
	\item QUERY\_STRING, REQUEST\_BODY: raw data
	\item REQUEST\_HEADERS: Header Werte
	\item REQUEST\_METHOD: die Verwendete HTTP Methode (POST/GET)
	\item REQUEST\_URI
	\item REQUEST\_LINE
	\item REMOTE\_ADDRESS: die Adresse des zugreifenden Clients
\end{itemize}

\subsubsection{Operatoren}

Operatoren beginnen meistens mit einem @-Symbol, einige Beispiele:

\begin{itemize}
	\item "@contains <script >": überprüft ob in den Arguments ein script-Tag vorkommt
	\item "@detectSQLi": überprüft die Arguments auf SQL-Injection-Muster
	\item "@detectXSS": überprüft die Arguments auf XSS-Muster
	\item "@inspectFile /path/to/util/runav.pl": die Arguments werden an das angeführte Skripts übergeben, das Ergebnis des Operators ist das Ergebnis des Skripts. Auf diese Weise kann z. B. jedes hochgeladene File mittels eines Virenscanners überprüft werden, bevor dieses File an den Applikationsserver übergeben wird.
	\item "@ipMatch 192.168.1.100": Vergleich auf eine IP-Adresse
	\item "@validateDTD /path/to/xml.dtd": überprüft ob das übergebene Argument einem definierten XML-Schema entspricht.
\end{itemize}

\subsubsection{Actions}

Actions definieren die Operationen, die exekutiert werden sollten falls die Operatoren erfolgreich angewandt werden konnten. Diese Operatoren können lt. \textit{mod\_security} Dokumentation in verschiedene Bereiche eingeteilt werden:

\begin{itemize}
	\item disruptive: verändern den Datenfluss, z. B. \textit{allow} oder \textit{deny}
	\item flow: beziehen sich auf die Regeln selbst, z. B. \textit{skip} oder \textit{chain}
	\item meta-data: dienen zur Dokumentation
	\item variable: für komplexe Regeln können Variablen gesetzt bzw. ausgelesen werden
	\item logging: Definition der Daten, die geloggt werden sollten
	\item special
	\item miscellaneous: everything else
\end{itemize}

\subsubsection{Beispiele}

Ein einfaches primitives Beispiel:

\begin{minted}{text}
SecRule ARGS "@contains <script>" log,deny,status:404
\end{minted}

Diese Regeln verwendet als Datenquelle alle extrahierten Argumente (\textit{ARGS}) und überprüft, ob in diesen ein HTML script-Tag vorkommt. In dem Fall wird die Anfrage mit einem 404 Fehler blockiert und geloggt.

Eine komplexere Regel:

\begin{minted}{text}
SecRule REQUEST\_LINE|REQUEST\_HEADERS|REQUEST\_HEADERS\_NAMES "@contains () \{" "id:420008,phase:2,t:none,t:lowercase,deny,status:500,log,msg:'Malware expert - user-agent: Bash ENV Variable Injection Attack'"
\end{minted}

Diese Regel sucht in verschiedenen Datenquellen nach dem String \textit{() \{}. Dieser String beschreibt einen Bash-Funktionsaufruf und sollte in regulärem Verkehr nicht vorkommen. Falls dieser String gefunden wird, wird der Request mit einem 500er Statuscode abgelehnt und protokolliert.

\begin{minted}{text}
SecRule REMOTE\_ADDR "@ipMatch 192.168.1.101" id:102,phase:1,t:none,nolog,pass,ctl:ruleEngine:off
\end{minted}

Diese Regel würde Verkehr von der IP-Adresse 192.168.1.101 immer akzeptieren (durch ruleEngine:off werden keine Folgeregeln mehr verwendet), diese Requests werden ebenso nicht geloggt. Dadurch wird ein Whitelisting der IP-Adresse durchgeführt.

\begin{minted}{text}
SecRule ARGS:username “@streq admin” chain,deny
SecRule REMOTE\_ADDR “!streq 192.168.1.111”
\end{minted}

Das letzte Beispiel zeigt, wie zwei Regeln mittels \textit{chain} verknüpft werden. Es wird zuerst die erste Regel angewendet (\textit{für das Argument username wurde der Wert admin verwendet}) und dann mit der Regel \textit{die IP-Adresse ist nicht 192.168.1.111} kombiniert. Falls beide Regeln wahr sind, wird die Anfrage abgelehnt. Dies entspricht also der Regel \textit{Ein Administrator darf sich nur ausgehend von der IP-Adresse 192.168.1.111 einloggen}.

\subsection{Log format}

Das \textit{mod\_security} Format entspricht nicht 100\% anderen bekannten Log-Formaten. Eine protokollierte Transaktion wird über mehrere Sublogs beschrieben. Jedes Sublog beginnt mit einem MIME-ähnlichem Header, dieser besteht aus einer eindeutigen ID über welchen die Sublogs aggregiert werden können (in dem Beispiel \textit{f9adec1d}) und einer Log-Kategorie (dies ist der angehängte Buchstabe).

\begin{minted}{text}
--f9adec1d-A--
[25/Sep/2016:18:42:50 +0300] V@fwen8AAQEAAA2TDbQAAAAK 127.0.0.1 35965 127.0.0.1 443
--f9adec1d-B--
GET / HTTP/1.1
Host: malware.expert
Accept: */*
User-Agent: () {

--f9adec1d-F--
HTTP/1.1 500 Internal Server Error
Content-Length: 613
Connection: close
Content-Type: text/html; charset=iso-8859-1

--f9adec1d-E--

--f9adec1d-H--
Message: Access denied with code 500 (phase 2). String match "() {" at REQUEST_HEADERS:User-Agent. [file "/etc/modsecurity/malware_expert.conf"] [line "97"] [id "420008"] [msg "Malware expert - user-agent: Bash ENV Variable Injection Attack"]
Action: Intercepted (phase 2)
Stopwatch: 1474818170070960 1010 (- - -)
Stopwatch2: 1474818170070960 1010; combined=177, p1=25, p2=147, p3=0, p4=0, p5=5, sr=55, sw=0, l=0, gc=0
Response-Body-Transformed: Dechunked
Producer: ModSecurity for Apache/2.9.0 (http://www.modsecurity.org/).
Server: Apache
Engine-Mode: "ENABLED"

--f9adec1d-Z--
\end{minted}

In Abhängigkeit der Transkation können unterschiedliche Log-Kategorien befüllt werden. \textit{A} beinhaltet immer den Zeitstempel und die Adressen der betroffenen Kommunikationspartner, \textit{B} ist der eingehende Request, \textit{F} die erzeugte Antwort; \textit{H} beschreibt die Regel die angewandt wurde.

Leider ist es nicht so einfach diese Informationen mittels \textit{grep} zu parsen.

\section{OWASP Core Ruleset}

Es macht natürlich wenig Sinn, für jede \textit{mod\_security}-Installation getrennte Regelwerke für common Angriffsvektoren zu erstellen. Aus diesem Grund haben sich Default-Regelwerke gebildet die als Initiale Regeln eingespielt werden können. Das bekannteste Beispiel hierfür ist das \textit{OWASP Core Rule Set} (auch CRS genannt), die aktuelle Version ist 3, dieses Regelwerk wird gerne zum Hardening von Webservern verwendet.

Dieses Ruleset beinhaltet Regeln welche die Angriffsvektoren der OWASP Top 10 (als auch weitere Angriffsvektoren) abdecken. Um dieses Regelwerk zu aktivieren, lädt der Administrator den Regelsatz als auch ein Konfigurations-File (\textit{crs-setup.conf}) vom OWASP-Server herunter und inkludiert diese innerhalb der Konfiguration des Apache Webservers. Dabei kann der Admin entweder alle Regeln inkludieren oder nur einzelne Subbereiche (z. B. nur Regeln für SQL-Injection Angriffe).

Ein häufiges Problem von Web Application Firewalls ist, dass valide Anfragen als Schadcode erkannt werden (diese Fehler werden auch als \textit{false positives} bezeichnet). Eine hohe Rate an false positives zerstört die Benutzerakzeptanz und führt im Normalfall dazu, das die WAF aus Akzeptanzgründen wieder deaktiviert wird. Um dem entgegenzuwirken besitzt das OWASP Core Rule Set eine Paranoia-Einstellung welche von 1 (default) bis 4 gesetzt werden kann. Je höher das Paranoia-Level umso mehr Schadcodes werden erkannt, allerdings steigt damit auch das Risiko von False-Positives. In Abhängigkeit der Gefährdung der Applikation kann nun ein geeignetes Paranoia-Level gewählt werden.

\section{Reflektionsfragen}

\begin{enumerate}
	\item Was versteht man unter einer Web-Application Firewall? Welche Einsatzzwecke bzw. Verwendungsmöglichkeitne gibt es für solche?
	\item Aus welchen Teilen besteht eine mod\_security Regel? Erläutere die jweiligen Teile.
	\item Was wird durch folgende Regel: \textit{SecRule ARGS "@contains <script>" log,deny,status:404} durchgeführt?
\end{enumerate}

\begin{enumerate}
	\item Was ist der Grundgedanke dabei, DevOps und Security zu verbinden?
	\item Welche Sicherheitsmaßnahmen können im Zuge von SecDevOps automatisiert durchgeführt werden? Erläutere die jeweiligen Maßnahmen.
	\item Was sind Docker Registries? Welchen negativen Einfluss auf die Sicherheit können diese nehmen?
	\item Auf welche Best-Practises sollte beim Bau von Container geachtet werden?
	\item Welche grundlegenden Sicherheitsprobleme gibt es bei Docker (bzw. bei Containerlösungen)? Erläutere vier davon.
\end{enumerate}
