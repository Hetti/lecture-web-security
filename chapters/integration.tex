\chapter{Integration mit der Umgebung}

Eine Webapplikation sollte niemals isoliert betrachtet werden. Auch wenn die vorgestellten Komponenten perfekt umgesetzt werden, ergibt sich aus der Interaktion zwischen der theoretischen Webapplikation und der realen Umgebung immer ein Gefahren- bzw. Verbesserungspotential.

Hierbei kann es sich z.B. um nicht-funktionale Elemente wie Logging handeln, oder aber auch um Aspekte wie Programmiersprache-inhärente Muster, die einen negativen Einfluss auf die Sicherheit der Webapplikation besitzen können.

\section{Using Components with Known Vulnerabilities}

Die Verwendung vorhandener (und gewarteter) externer Komponenten wie Bibliotheken oder Frameworks besitzt sicherheitstechnisch viele Vorteile: man kann von Fehlern anderen lernen (und vermeidet diese selbst) bzw. muss das Rad nicht neu erfinden.

Damit wird allerdings auch der Nachteil eingekauft, dass eine Verwundbarkeit innerhalb einer integrierten externen Komponente automatisch auch eine Verwundbarkeit der eigenen Applikation impliziert. Daher müssen aktiv und regelmäßig verwendete Komponenten auf bekannte Schwachstellen hin überprüft, und ggf. die betroffenen Komponenten aktualisiert werden.

Um den Aufwand dieser Überprüfungen zu reduzieren und damit optimaler-weise deren Frequenz zu erhöhen gibt es automatisierte Tools wie den OWASP Dependency-Check\footnote{\url{https://jeremylong.github.io/DependencyCheck/analyzers/index.html}}. Diese analysieren automatisch Projekte auf Dependencies (z.B. über Ruby Gemfiles, NPM package-lock.json Files, Maven Projektbeschreibungen), extrahieren automatisch dependencies und korrelieren diese mit öffentlichen Verwundbarkeitsdatenbanken. Wird hier nun eine potentiell anwendbare Schwachstelle gefunden, wird der Entwickler via Email oder Slack notifiziert.

\section{Insufficient Logging and Monitoring}

Diese Schwachstelle wurde im Jahre 2017 neu bei den OWASP Top 10 aufgenommen. Es handelt sich hierbei weniger um eine Schwachstelle während der Exekution, sondern eher um die Schaffung der Möglichkeit nach einem Angriff aufgrund der vorhandenen Log-Dateien das Vorgehen des Angreifers und die betroffenen Daten zu erkennen.

Folgende groben Anforderungen an das Log-System werden gestellt:

\begin{itemize}
	\item Es muss mit verteilten Applikationen umgehen können. Eine Webapplikation ist potentiell auf mehrere Computersysteme verteilt (Webserver, Applikationsserver, Datenbankserver). Die Logdaten der gesamten Systeme sollten an einer Stelle aggregiert werden.
	\item Es muss die Integrität der Logdaten schützen: ein Angreifer sollte keine Möglichkeit besitzen, die geloggten Daten zu beeinflussen. Würden z.B. Logdaten direkt am Webserver gespeichert werden, könnte ein Angreifer der den Webserver gehackt hat, ebenso die Logdaten modifizieren. Dies impliziert, dass der Log-Server über eine genau definierte API erreichbar sein sollte.
	\item Es muss die Vertraulichkeit der Daten schützen. Da der Logserver Detailinformationen über betriebliche Abläufe speichert, müssen diese Daten mindestens ebenso sicher wie die ursprünglichen Daten gespeichert werden.
	\item Das Log-System muss Möglichkeiten zur nachträglichen Auswertung der gesammelten Daten bieten. Bonuspunkte, wenn man ein automatisiertes Monitoring mit dem Log-System betreiben kann.
\end{itemize}

Die jeweiligen loggenden Systeme sollten alle sicherheitsrelevanten Events (z.B. Input Validation Fehler, Authentication Fehler, Authorization Fehler, Applikations-Fehler) an das zentrale Log-System schicken. Diese Daten sollen um Business Process Events angereichert werden. Diese dienen dazu, relevante geschäfts-relevante Prozesse und Ereignisse mit den Sicherheits-Events zu korrelieren. Weitere Datenquellen sind z.B. Anti-Automatisierungssysteme, welche Brute-Force Angriffe erkennen, Datenverarbeitungssysteme (können auch Batch-Systeme sein, die z.B. einen Daten-Export oder Backups ausführen) und alle direkt und indirekt involvierten Services, wie z.B. Mailserver, Datenbankserver, Backupdienste. Falls vorhanden, sollten die Loginformationen sicherheitsrelevanter Komponenten (HIDS, NIDS, WAFs) auf jeden Fall inkludiert werden.

Bei dem Loggen sollte darauf geachtet werden, dass, wenn möglich, standardisierte Log-Formate wie CEF oder LEEF verwendet werden. Dadurch wird das Konvertieren der jeweiligen Datenquellen auf ein gemeinsames Format vermieden.

Welche Daten sollten pro Event erfasst werden?

\begin{itemize}
	\item Wann hat sich der Vorfall ereignet? Bei einer verteilten Applikation sollte hier darauf geachtet werden, dass Timestamps die Zeitzone beinhalten (und auch auf Zeitumstellungen achten). Grundlage für das temporale korrelieren von Events ist es, dass alle beteiligten Server eine idente Systemzeit besitzen (z.B. durch die Verwendung von ntp).
	\item Wo ist das Event passiert? Hierfür können Systemnamen, Servicenamen, Containernamen oder Applikationsnamen verwendet werden.
	\item Für welchen Benutzer ist das Event passiert? Hier können Systembenutzer (mit denen das Service läuft) oder feingranular der gerade eingeloggte Benutzer protokolliert werden.
	\item Was ist passiert? Dies wird immer applikations- und event-spezifisch sein. Viele Systeme verwenden zumindest eine idente Klassifizierung der Wichtigkeit des Events.
\end{itemize}

Die Verwendung von personenbezogenen Daten kann das Logging verkomplizieren. Ein Unternehmen sollte klare Regeln erstellen, welche Daten geloggt werden und, falls notwendig, Anonymisierung oder Pseudonymisierung verwenden um sensible Daten zu maskieren. Ein ähnliches Problem tritt auf, wenn Log-Informationen zwischen Unternehmen geteilt werden sollte (z.B. im Zuge eines Informations-Lagebilds). Da diese Daten unter anderem personen-bezogene Informationen als auch Betriebsgeheimnisse inkludieren können, wird davon meistens abgesehen.

Die erfassten Daten sollten im Zuge einer Auswertung verwendet werden. Hier werden häufig "normale" Texteditoren in Verbindung mit regulären Ausdrücken verwendet. Fortgeschrittene Lösungen wären ELK-Stacks, Kibana und Logstash und z.B. Splunk.

In einem ähnlichem Umfeld arbeiten SIEM-Systeme (Security Information and Event Management). Diese werden zumeist als weiterer Schritt nach Log-Management angesehen. Zusätzlich zum Log-Management wird zumeist auch Security Event Management (real-time monitoring), Security Information Management (long-term storage of security events) und Security Event Correlattion durchgeführt.

\section{DevOps und Tooling}

DevOps ist eine neuere Strömung die versucht, Development und Operations zu vereinen.

Webapplikationen werden zumeist von Entwickler erstellt und dann einem Administratoren-Team zur Installation übergeben. Teilweise wird die Applikation auch von den Entwicklern installiert und dann von den Administratoren langfristig gewartet. In größeren Unternehmen wird die Installation und Wartung teilweise auf zwei unterschiedliche Administratorenteams aufgeteilt.

Dies führt zu getrennten Teams mit getrennten Wissensstand und kann im worst-case auch z.B.nkerdenken --- \textit{,,us vs. them''}--- führen. Diese Trennung behindert den Informationsfluss und verhindert, und führt künstliche Schranken im Verantwortlichkeitsgefühl ein (,,die Admins sind dafür verantwortlich''). Im Fehlerfall führt dieses Bunkerdenken auch zum Herum schieben der Verantwortung zwischen Parteien.

DevOps versucht nun, wie der Name schon sagt, die Trennung von Entwicklung (,,Development'') und Administration (,,Operations'') zu beenden. Prinzipiell ist DevOps mehr eine Philosophie/gelebte Firmenkultur die stark von der Kultur der kontinuierlichen Verbesserung (z.B. Kanban in Japan) geprägt ist. Bei der Umsetzung bindet es stärker Entwickler in klassische Operations-Bereiche wie Deployment ein.

Eine gute Beschreibung ist, dass DevOps agile Entwicklungsmethoden mit agilen Deployment kombiniert.

\subsection{Agile Methoden}

Agile Methoden sind ein neueres Projektmanagement-Muster, welches im Agile Manifesto\footnote{\url{https://agilemanifesto.org/}} folgende Grundsätze definiert:

\begin{itemize}
	\item Individuals and interactions over processes and tools
	\item Working Software over comprehensive documentation
	\item Customer collaboration over contract negotiation
	\item Responding to change over following a plan
\end{itemize}

Umgesetzt führt dies zumeist dazu, dass monolithische Projekte in kleine minimale Teile transformiert werden. Diese werden dann, in Reihenfolge der Kundenprioitisierung, abgearbeitet und regelmäßig der Fortschritt mit dem Kunden besprochen. Im Zuge des Projektes kommt es häufiger zu Änderungswünschen, diese können dann als weiteres Teilprojekt/Schritt inkludiert werden, die Geschwindigkeit des Teams kann über Projektdauer immer genauer eingeschätzt werden.

Damit die Projektsteuerung bei agiler Methodik funktioniert, darf ein einmal erledigter Schritt/Problem nicht immer wieder (durch Bugs) kosten verursachen. Aus diesem Grund wird hier stark auf automatisierte Tests gesetzt. Sofern diese Tests erfolgreich durchlaufen wird davon ausgegangen, dass das Produkt funktioniert. Der \textit{master} oder \textit{production}-Branch der Software sollte niemals fehlerhafte Testcases besitzen, kann daher jederzeit an den Kunden ausgeliefert werden.

\subsubsection{Anwendbarkeit agiler Methoden}

Agile Methoden sind natürlich nicht für alle Projekte geeignet und werden eher bei Startup-Projekten bzw. explorativen Projekten angetroffen. Bei Anwendungen mit hoher Sicherheitsrelevanz gibt es Zeitweise sehr genau ausspezifizierte Lasten-/Pflichtenhefte, diese müssen dann auch dementsprechend umgesetzt werden.

\subsection{Infrastructure as Code}

Ein Grundsatz Agiler Methoden ist ,,\textit{Working Software over comprehensive documentation}''.

Der Fokus auf \textit{Working Software} anstatt auf Dokumentation schlägt sich auch bei dem Deployment (dem Installieren der Software) nieder: dies wird zumeist automatisiert als Skript durchgeführt und nicht als Dokumentation ausgeliefert (und entspricht dadurch bereits dem DevOps-Gedanken).

Da im Zuge von Agilen Methoden versucht wird möglichst früh und möglichst häufig lauffähigen Code beim Kunden bereitzustellen (bzw. als Testservice dem Kunden zur Verfügung zu stellen) passieren Installationsvorgänge regelmäßig. Um hier nun Redundanzen zu vermeiden (bzw. um Konfigurationsfehler zu verhindern) wurden hier (historisch betrachtet) Installationsanweisungen immer stärker durch automatisierte Skripts ersetzt. Danach wurden dezidierte Deploymentstools (wie z.B. \textit{capistrano}) für das Setup der Applikation konfiguriert und verwendet. Im Laufe der Zeit wurden diese Tools nicht nur für die Applikation selbst, sondern auch für Datenbanken, Systemservices, etc. angewandt; die historische Evolution sind mittlerweile dezidierte Frameworks die zum Setup der Systeme dienen (wie z.B. Puppet, Chef oder Ansible).

Ein weiterer Vorteil dieses Ansatz ist, dass die verwendete Konfiguration innerhalb der (hoffentlich) verwendeten Source Code Versionierung automatisch versioniert und Veränderungen dokumentiert werden. Dies erlaubt das einfachere Debuggen von Regressionen.

\subsection{Continuous Integration and Continuous Delivery}

Die Verwendung von Tests zur Sicherung der Softwarequalität ist ein essentieller Bestandteil Agiler Methoden. Dem Kunden werden regelmäßig neue Programmversionen mit erweiterten Features zugestellt und anhand des Kundenwunsches neue Features selektiert und im nächsten Programmier-Sprint hinzugefügt. Würden Features fertig gestellt werden, die Fehler beinhalten und müsste man nachträglich diese Fehler immer wieder neu korrigieren, würde dadurch die Geschwindigkeit des Programmierteams stark leiden. Um dies zu verhindern, werden massiv Softwaretests geschrieben, die überprüfen ob die gewünschten Kundenfeatures ausreichend implementiert wurden. Diese Tests werden aufgerufen, bevor ein neues Feature in die, dem Kunden übermittelten, Version integriert wird. Auf diese Weise wird automatisch eine Kontrolle der Qualität durchgeführt.

Um sicher zu stellen, dass diese Tests auch wirklich aufgerufen werden, werden diese Tests automatisiert aufgerufen. Dies wird im Zuge des Continuous Integration Prozess durchgeführt: nach jeder Änderung wird versucht, die Software zu bauen (builden) und anschließend werden die vorhandenen Tests ausgeführt. Falls ein Fehler auftritt, wird der betroffene Entwickler sofort notifiziert.

Während Unit-Tests primär auf das Testen von Funktionen abzielen, werden auch statische Source Code Tests integriert. Diese überprüfen die Qualität des gelieferten Codes (z.B. Coding Guidelines) und sind n zweites Standbein automatisierter Tests.

In einem finalen Schritt kann auch Continuous Delivery angewandt werden, dies ist quasi die Ausdehnung des Continuous Integration Prozesses auf das installieren der fertigen Software (in einer Testumgebung oder, im Extremfall, direkt beim Kunden). Hier kann nach erfolgtem Bauen und Testen der Software diese auf Knopfdruck (oder vollautomatisiert) in einem Test- bzw. Produktivsystem eingespielt werden. Der Fluss von der Entwicklung über die Kontrolle bis zur Installation ist somit vollzogen. Falls dies alles durch den Entwickler konfiguriert wurde, kommt kein klassischer System-Administrator im Prozess vor. Damit gibt es keine Teilung der Kompetenzen und Verantwortung mehr, der Schritt zu DevOps ist vollzogen. 

\section{DevOps and Security}

Die Abkürzung DevOps besteht aus Development und Operations, das Wort Security kommt dabei nicht vor.

Sicherheit in DevOps zu integrieren ist ein ähnliches Problemfeld wie das ursprüngliche DevOps-Problem der Trennung zwischen Admins und Entwicklern. Wenn die Sicherheitsverantwortlichen ein getrenntes Team sind, dann ergibt sich wieder Bunkerdenken, Wissensinseln als auch ein fehlendes Verantwortungsgefühl.

Es wird nun Versucht das Sicherheitsteam in das DevOps-Team zu integrieren und dadurch Security als gemeinsame Verantwortung zu etablieren. Die Grundidee ist schön in folgendem Satz ausgeführt: ,,\textit{Security is built into the System instead of being applied upon the finished product}''.

Häufiger wird zwischen verschiedenen Ansätzen um Security einzubauen unterschieden:

\begin{itemize}
	\item DevOpsSec: es wird ein Produkt entwickelt, danach wird die Administration durchgeführt. Final wird Security gewährleistet: ein Beispiel dafür wäre es, dass nach Inbetriebnahme das Security-Team Security-Patches einspielt. Das klassische Beispiel wäre das Anpassen und der Betrieb von Standardsoftware.
	\item DevSecOps: zuerst wird entwickelt, danach wird Security betrachtet und danach die Administration fortgesetzt. Dieser Ansatz wird aktuell (2019) häufig gesehen und ist zumindest besser als die Security generell zu ignorieren. Ein Beispiel hierfür wäre es, die Inbetriebnahme von der erfolgreichen Durchführung einer Sicherheitsüberprüfung abhängig zu machen.
	\item SecDevOps: betrachtet initial die Security (z.B. schon während der Planung der Software). Dadurch durchdringt Security die gesamte Entwicklung als auch die Administration.
\end{itemize}

Security bewirkt meistens einen Mehraufwand für die Entwickler. Um diesen Mehraufwand zu begrenzen, wird auch hier (im DevOps-Spirit) stark auf Automatisierung gesetzt. So werden z.B. Sicherheitstests als automatisierte Testprogramme implementiert und während der Testphase innerhalb des Continuous Integration Prozesses ausgeführt. Dadurch werden die Sicherheitstests automatisch Teil des Abnahme-/Verifikationsprozess und durchdringt auf diese Weise die gesamte Entwicklung.

\subsection{Automatisierung}

Um die konsistente und regelmäßige Verwendung der Sicherheitstests zu enforcen wird die Ausführung der Sicherheitstests stark automatisiert. Dieses Kapitel führt häufig verwendete automatisierte Tests an:

\subsubsection{Automated Unit Tests}

Unit Tests sind minimale Tests die ein Feature verifizieren. Die meisten Software-Frameworks erlauben es, Tests auf Controller-Ebene zu schreiben für deren Ausführung die Applikation nicht als gesamtes gestartet werden muss.

Alternativ kann ein Unit-Test z.B. als einfaches Shellskript, Python-Requests2 skript oder als JUnit-Test unter Verwendung von Java-Bibliotheken durchgeführt werden. In dem Fall muss im Zuge der Tests ein Webserver gestartet werden, diese Tests sind daher zeitaufwendiger.

Integrations-Tests testen die Funktionalität des Gesamtsystems. Hierbei kann unter anderem auf Browser-basierte Frameworks wie Selenium oder Capybara zurückgegriffen werden. Durch diese wird ein oder mehrere Benutzer mittels eines virtuellen Browsers simuliert --- die Tests beschreiben die Benutzernavigation und -operationen innerhalb der Webseite. Hier kann man z.B. zwei Benutzer simulieren: ein Benutzer legt Daten an und ein zweiter Benutzer versucht (invalid) auf diese Daten zuzugreifen. Diese Tests sind um einiges Ressourcen- (da ein Webbrowser und Webserver benötigt wird) als auch zeitaufwendiger und werden daher zumeist nicht nach jeder Sourcecode-Änderung durchgeführt.

\subsubsection{Static Source Code Tests}

Analog zur Analyse der Qualität des Source Codes (während des normalen DevOps-Prozesses) können auch Tools zur statischen Analyse des Source Codes inkludiert werden. Diese prüfen den Source Code auf bekannte Programmiermuster und -fehler die sicherheitsrelevante Konsequenzen besitzen können.

Hier gibt es meistens Programmier- und Framework-abhängige Tools wie z.B. \textit{bandit} für \textit{Python}, \textit{Brakeman} für \textit{Ruby on Rails} und \textit{SpotBugs} für \textit{Java}. Eine Ebene über diesen Einzeltools funktioniert \textit{OWASP SonarCube}. Dieses Tool kann intern die gesamten erwähnten Subtools anwenden und besitzt auch Plugins um mittels \textit{OWASP dependency-check} eine Überprüfung der verwendeten Abhängigkeiten (z.B. Bibliotheken) auf Schadcode hin durchzuführen.

\subsubsection{Dynamic Application Scans}

Zusätzlich zur statischen Source-Code Analyse kann man auch dynamische Scans verwenden. Hierbei wird zumeist die Applikation in einer Testumgebung (z.B. \textit{staging}) automatisiert installiert und danach mittels automatisierter Web Application Security Scanner gescripted ein Test durchgeführt. Im Falle einer Regression werden die Entwickler benachrichtigt. Beispiele hierfür wäre z.B. das automatisierte Scannen einer Applikation mittels \textit{OWASP ZAP} unter Zuhilfename des \textit{full-scan.py}-Skripts.

\section{Reflektionsfragen}

\begin{enumerate}
	\item Was ist der Grundgedanke dabei, DevOps und Security zu verbinden?
	\item Welche Sicherheitsmaßnahmen können im Zuge von SecDevOps automatisiert durchgeführt werden? Erläutere die jeweiligen Maßnahmen.
\end{enumerate}
