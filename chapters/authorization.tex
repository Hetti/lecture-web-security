\chapter{Authorization}

Unter Authorization versteht man die Kontrolle der Benutzerrechte vor der Ausführung der jeweiligen Operation. Die Authorization muss immer server-seitig und zum Zeitpunkt des Operationsaufruf passieren. Authorization benötigt eine zuvorige Authentication des Benutzers.

\section{Probleme während der Authorization}

Die Problematik fehlender Authorization ist sehr ähnlich zu der Problematik fehlender Authentication. Ein großer Unterschied liegt bei der Eingrenzung der Auswirkung nach einem erfolgreichem Angriff. Bei einer fehlenden Authentication ist kein Benutzerlogin zur Ausführung von Operationen notwendig. Es kann daher jeder anonyme Internetbenutzer auf die Daten zugreifen, im Worst-Case werden sensible Daten auch von automatisierten Crawlern erfasst und gespeichert.

Bei einer fehlerhaften Authorization muss ein Angreifer sich zumindest authentifizieren. Im Fehlerfall kann der Betreiben den Login- und Registrierungsprozess unterbinden, alle Benutzer ausloggen und erhält auf diese Weise eine Plattform, auf die er trusted User wieder zulassen kann um einen eingeschränkten Betrieb zu ermöglichen (während der grundlegende Fehler behoben wird). Bei einer fehlenden Authentication muss die Plattform deaktiviert werden, da es diese Möglichkeit zur Einschränkung auf vertrauenswürdige Benutzer entfällt.

\subsection{Keine/Fehlende Authorization}

Die Applikation überprüft zwar die Authentication aber jeder eingeloggte Benutzer darf alle Operationen aufrufen.

Da ein Benutzer sich initial authentifizieren muss, kann aufgrund des Logverlaufs der Fehler zumindest analysiert und der bösartige User ausgesperrt werden. Eine Gefährdung durch Crawler findet ebenso nicht statt, da diese keine valide Authentication durchführen können.

Analog zu den Fehlern bei der Authentication, kann es zu Problemen kommen wenn eine Applikation innerhalb mehrere Programmiersprachen/Frameworks implementiert wurde bzw. weitere Komponenten zu einer bestehenden Lösung hinzugefügt wurden. Hauptproblem ist die Integration der Authorization in das Bestandssystem. Als Pen-Tester sollte man immer diese Komponenten gezielt auf Authentication- und Authorization-Fehler hin testen.

\textit{There is a special place in hell for developers that think that not-displaying UI elements is a kind of authorization}. Häufig werden in Abhängigkeit der Benutzerrolle nur Teile der Funktionalität angezeigt. Ein Angreifer der einen zweiten Benutzeraccount mit diesen Rechten (oder Log-Dateien) besitzt, erhält allerdings Informationen über die Operationen und kann diese direkt aufrufen. Dies ist ein Fall von Security by Obscurity.

Das erzwungene Aufrufen von Webseiten über URLs wird auch \textit{Forceful Browsing} genannt. Wird direkt auf Ressourcen, wie z. b. Downloadlinks, zugegriffen, wird dies \textit{direct object reference} genannt.

\subsection{Unterschiedliche Authorization in Alternate Channels}

Falls eine Webapplikation Daten oder Operationen auf unterschiedliche Arten und Weisen anbietet, müssen die Schutzmechanismen auf den verschiedenen Kanälen synchronisiert werden.

Wird z. B. eine Operation direkt mittels der Webseite als auch über eine SOAP Webservice-Schnittstelle (z. B. für eine mobile Applikation) angeboten, müssen auf beiden Schnittstellen die gleichen Zugriffsberechtigungen überprüft werden. Häufig kann man während Penetration-Tests verminderte Schutzmaßnahmen bei Webservice-Schnittstellen feststellen.

\subsection{Hint: Update Operationen}

Anhand eines Beispiels soll gezeigt werden, warum auch einfache Operationen komplexe Sicherheitsfragen aufwerfen können. Bei dem konkreten Beispiel sollen Benutzerdaten aktualisiert werden. Hierfür wird folgende Update-Operation aufgerufen:

\begin{minted}{HTTP}
POST /user/update/1 HTTP/1.1
\end{minted}

Als Parameter wird ein JSON-String mit den neuen Werten übergeben:

\begin{minted}{json}
{
	"id" : "1",
	"name" : "happe"
}
\end{minted}

Bei diesem Beispiel fallen folgende sicherheitsrelevanten Fragen an:

\begin{itemize}
	\item kann ich durch Setzen einer anderen ID (statt 1) in der URL auf einen anderen Datensatz schreibend zugreifen?
	\item was passiert, wenn man die ID im Datensatz ändert? Teilweise überprüfen Webapplikationen nur die ID innerhalb der URL und ignorieren die IDs innerhalb des Datensatzes. Mit Glück kann man diesen Missmatch zum Überschreiben anderer Datensätze verwenden.
	\item was passiert, wenn im Datensatz keine ID vorkommt und der Angreifer manuell ein ID-Element in das JSON-Dokument hinzufügt?
	\item was passiert, wenn der Angreifer ein neues JSON-Element namens \textit{``Admin'': ``true''} hinzufügt?
	\item was passiert, wenn der Angreifer statt HTTP POST eine HTTP GET Operation verwendet? HTTP GET sollte eigentlich eine read-only Operation sein, deswegen werden GET requests teilweise von Web-Application Firewalls nicht kontrolliert und man kann auf diese Weise Firewall-Regeln umgehen.
\end{itemize}

\subsection{Probleme mit Mass-Assignment}

Moderne Web-Frameworks versuchen die Effizienz von Programmierern zu verbessern. Ein Feature, welches potentiell negativen Einfluss auf die Sicherheit einer Applikation besitzt ist \textit{mass assignments}.

Unter Mass-Assignment versteht man das automatisierte Zuweisen von Werten aus einem HTTP Request zu einem Datenbank-Objekt. So könnten z. B. bei einem User-Update die übergebenen HTTP-Parameter automatisch gegenüber den bekannten Datenbankfeldern gematched werden und Parameter wie z. B. \textit{vorname} oder \textit{nachname} werden automatisch auf das Datenbankfeld \textit{vorname} und \textit{nachname} des betroffenen Datensatzes gemapped und aktualisiert.

In Ruby on Rails würde der betroffene Code folgendermaßen aussehen:

\begin{minted}{ruby}
	@user = User.find(params[:id])
	@user.update(params[:user])
\end{minted}

Die erste Zeile des Beispiels verwendet den übergebenen \textit{id} Parameter um aus der Datenbank ein User-Objekt zu laden. In der zweiten Zeile werden nun die vorhandenen HTTP-Parameter automatisch den vorhandenen Feldern des User-Objektes zugewiesen.

Dies ist aus Sicherheitssicht problematisch. Ein Angreifer kann versuchen potentielle Datenbankfelder zu erraten und diese mittels mass-assignment zuzuweisen. Beispiele wären z. b. das Setzen von \textit{role=admin} oder \textit{admin=true}.

Um dies zu verhindern besitzen die meisten Frameworks ein Möglichkeit Attribute für das Mass-Assignmentexplizit zu verbieten (black-list) oder zu erlauben (white-list). Aus Sicherheitssicht ist das automatische Ablehnen von Attributen und die explizite Freigabe einzelner Attribute (also das White-Listing) vorzuziehen.

In Ruby on Rails würde der betroffene Code folgendermaßen aussehen:

\begin{minted}{ruby}
	@user = User.find(params[:id])
	@user.update(params.require(:user).permit(:full_name))
\end{minted}

In diesem Fall werden nur die Felder \textit{full\_name} für das Objektes \textit{user} mittels mass-assignment aktualisiert.

\subsection{Scoping von Daten}

Unter Scoping von Daten versteht man das Einschränken der verfügbaren Daten auf einen Subbereich. Eine Web-Operation wird meistens für einen Benutzer aufgerufen, die potentiell verfügbaren Daten sollten so früh wie möglich auf die für den Benutzer verfügbaren Daten eingeschränkt werden.

Beispiel: eine Applikation verwaltet Rechnungen, jede Rechnung hat einen Benutzer als Autor. Mittels einer Update-Operation kann eine Rechnung bearbeitet werden. Dies geschieht mittels der Operation \url{/invoice/1/update}, in der Applikation ist der gerade angemeldete autorisierte User als \textit{current\_user} bekannt, mittels \textit{current\_user.invoices} kann man auf die Rechnungen des aktuellen Users zugreifen, mittels \textit{Invoice} auf alle Rechnungen die dem System bekannt sind.

Die Update-Operation sollte nun folgendermaßen aussehen:

\begin{minted}{ruby}
	# hier sollte NICHT Invoice.find(params[:id]) stehen
	@invoice = current_user.invoices.find(params[:id])

	# normaler Update-Code
	@invoice.update(the_data_which_will_be_updated)
	@invoice.save
\end{minted}

Durch die Verwendung des User-Scopes wird implizit der Zugriff auf Rechnungen des aktuellen Benutzers erzwungen. Dadurch müssen Zugriffsberechtigungen nicht zusätzlich explizit kontrolliert werden.

\section{Probleme bei Verwendung von WebSockets} 

WebSockets unterliegen nicht der Same-Origin-Policy moderner Webbrowser. Es wird daher empfohlen, bei öffnen des WebSockets serverseitig den \textit{Origin}-Header auf valide aufrufende Webseiten hin zu überprüfen. Während des initialen Handshakes wird ein \textit{Web-Socket-Key} übertragen. Dieser dient nur zur Identifikation des Browsers gegenüber dem Webserver und darf nicht zu Authentications- bzw. Authorization-Zwecken verwendet werden.

Über WebSockets werden zumeist Nachrichten verschickt. Der Server wird auf den Erhalt einer Nachricht hin eine Operation starten, vor dieser muss der Server eine Berechtigungsüberprüfung durchführen. Die Berechtigungen zwischen WebSockets und HTTP-basierte Kommunikation müssen auf jeden Fall synchron gehalten werden (siehe auch, Different Authorization in Alternate Channels).  
Ein häufiges Problem ist, dass der Client vor der Authentication des WebBrowsers gegenüber dem Client bereits einen WebSocket öffnet. In diesem Fall wird meistens eine parallele Session-Verwaltung auf server-seite aufgebaut: innerhalb einer Serverdatenbank wird für den WebSocket-Client eine Session-Id oder eine TokenId generiert, in der Datenbank der betreffende Web-Benutzer dem Token zugeordnet und dem Client das token/die session mitgeteilt. Der Webbrowser muss nun bei jeder WebSocket Anfrage dieses Secret mit übertragen und der Server kann den User über dieses Token identifizieren.

\section{Time of Check, Time of Use (TOCTOU)}

Der Zeitpunkt der Berechtigungsüberprüfung ist essentiall. Sogenannte \textit{Time of Check, Time of Use} Angriffe nutzen racing conditions zwischen der Überprüfung von Operationsberechtigungen und der Ausführung von Operationen aus.

Ein gutes Beispiel für die TOCTOU-Problematik sind zumeist Token-basierte Systeme. Hier wird die Berechtigung eines Anwenders überprüft und danach ein Token mit einer definierten Laufzeit ausgestellt. Wird das Token während der Laufzeit an eine Operation überreicht, wird dieses valide Token als Zugangsberechtigung akzeptiert und die Operation ausgeführt auch wenn zwischenzeitlich die Berechtigungen des Benutzer modifiziert wurden und der Benutzer die Operation eigentlich nicht mehr ausführen dürfe.

\section{Reflektionsfragen}

\begin{enumerate}
	\item Erkläre den Unterschied zwischen Identifikation, Authentication und Authorization?
	\item Was versteht man unter Authorization? Wann und wo sollte diese durchgeführt werden? Welches Sicherheitsproblem versteht man unter Insecure Direct Object Reference bzw. unter Forced Browsing?
	\item Welche Probleme können im Zusammenhang mit Mass-Assignment auftreten?
	\item Gegeben ein Webshop mit einem Downloadlink für Rechnungen \url{http://shop.local/invoices/1/download}. Welche sicherheitsrelevanten Fehler können hier nun auftreten?
	\item Gegeben eine Profil-Updatefunktion welche als POST \url{/user/1/update} implementiert wurde, als Parameter werden die Felder \textit{id}, \textit{email}, \textit{new\_password} und \textit{rolle} (mit Wert \textit{user}) übergeben. Erkläre zumindest drei Sicherheitsprobleme die während des Updates eines Benutzers auftreten können.
\end{enumerate}

