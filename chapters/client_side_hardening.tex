\chapter{Clientseitige Schutzmaßnahmen}

\section{Integration externer Komponenten}

Werden externe Inhalte innerhalb der eigenen Seite inkludiert, erhöht sich die Angriffsfläche: ein Angreifer mit Zugriff auf die externe Seite kann über diese Schadcode in die, eigentlich sichere, eigene Webseite einschleusen. Falls die Einbindung externer Inhalte zwingend benötigt wird, kann das Gefahrenpotential durch Einsatz folgender Techniken reduziert werden:

\subsection{IFrame: sandbox-Flag}
\index{IFrame Sandbox Attribute}

Wird eine externe Resource über das HTML \textit{iframe} Tag eingebunden kann durch Verwendung des \textit{sandbox}-Attributes die Sicherheit erhöht werden. Bei Verwendung dieses Attributes werden folgende Einschränkungen aktiviert:

\begin{itemize}
\item JavaScript wird für die eingebundene Resource deaktiviert.
\item Die eingebundene Seite bekommt einen eigenen Origin; dadurch kann dieses nicht mehr auf die einbindende Seite zugreifen, auch wenn diese auf dem identen Server abgelegt waren.
\item Die eingebundene Seite kann keine neuen Fenster bzw. Dialoge öffnen.
\item Es können keine Formulare abgeschickt werden.
\item Plugins werden für die eingebettete Seite deaktiviert.
\item Autoplay wird deaktiviert.
\end{itemize}

Die Einschränkungen können durch mehrere Optionen aufgeweicht werden, der Namen dieser Optionen beginnt mit \textit{allow-}. Ein abschließendes Beispiel für einen per Iframe eingebundenen Twitter-Button:

\begin{minted}{html}
<iframe sandbox="allow-same-origin allow-scripts allow-popups allow-forms"
src="https://platform.twitter.com/widgets/tweet_button.html"
style="border: 0; width:130px; height:20px;"
</iframe>
\end{minted}

\subsection{Subresource Integrity (SRI)}
\index{SRI}\index{Subresource Integrity}

Webapplikationen lagern statische Dateien häufig auf externe Server aus. Ein Beispiel hierfür wäre z.B. das Auslagern von statische Javascript- oder CSS-Dateien auf ein CDN-Netzwerk. Dies wird zumeist zur Erhöhung der Performance bzw. Reduktion der Latenzzeit durchgeführt.

Ein Angreifer, der Zugriff auf einen externen Server erlangt, kann auf diesen bösartigen JavaScript- oder CSS-Code hinterlegen. Lädt nun eine Webseite diese Dateien, wird diese automatisch infiziert. Um diesen Angriffsvektor zu vermeiden, kann Subresource Integrity verwendet werden. Bei dieser Technik wird ein Hashwert für jede eingebundene Resource berechnet und innerhalb der eigenen Webseite angegeben. Wird nun eine externe Resource angefordert, berechnet der Webbrowser den Hashwert der empfangenen Resource und vergleicht diesen mit dem konfigurierten Hashwert (innerhalb der Webseite). Die Resource wird nur verwendet, wenn diese Hashwerte ident sind.

Ein Beispiel:

\begin{minted}{html}
<script src="https://example.com/example-framework.js"
integrity="sha384-oqVuAfXRKap7fdgcCY5uykM6+R9GqQ8K/uxy9rx7HNQlGYl1kPzQho1wx4JwY8wC"
crossorigin="anonymous">
</script>
\end{minted}

Mittels CSP kann die Verwendung von Subresource Integrity erzwungen werden, folgendes Beispiel erzwingt die Angabe von Hashsummen für alle inkludierte JavaScript- und CSS-Dateien.

\begin{minted}{text}
Content-Security-Policy: require-sri-for script style;
\end{minted}


\section{Referrer-Policy}
\label{referrer_policy}
\index{Referrer-Policy}
\index{HTTP Header!Referrer-Policy}

Ein Standard-Header der von Webbrowsern gesetzt wird ist der \textit{Referer}-Header. Dieser inkludiert bei jedem Seitenaufruf die URL der aufrufenden Seite. Dies kann einen negativen Security-Impact haben, falls die URL der aufrufenden Seite sensible Informationen (wie z.B. eine Session-Id oder auch sensible Benutzerdaten) beinhaltet. Potentiell wird der Header vom Empfangsserver und, bei unverschlüsselter Kommunikation, von allen verbundenen Geräten entlang des Kommunikationspfades gesehen.

Web-Server können das gewünschte Verhalten durch Verwendung des Headers \textit{Referrer-Policy}\footnote{Achtung: während der Header aufgrund eines Rechtschreibfehlers \textit{Referer} heißt, heißt der Policy Header \textit{Referrer-Policy}.} mitteilen, valide Werte sind:

\begin{description}
	\item[no-referrer]: der Referer-Header wird nicht übertragen.
	\item[no-referrer-when-downgrad]: die URL wird als Referer übertragen sofern die Folgeseite nicht ein unsichereres Protokoll verwendet. Dadurch wird z.B. eine Übertragung der URL beim Übergang von HTTPS zu HTTP verboten. Dies ist häufig das Default-Verhalten der Webbrowser.
	\item[origin]: es wird immer nur der Origin übertragen.
	\item[origin-when-cross-origin]: es wird die volle URL übertragen, sofern man sich innerhalb des identen Origins befindet. Falls es zu einem Origin-Wechsel kommt (z.B. durch Navigation auf eine externe Seite) wird nur der Origin übertragen.
	\item[same-origin]: solange man sich innerhalb des Origins befindet, wird die URL gesetzt, ansonsten wird kein Referer-Header versendet.
	\item[strict-origin]: es wird immer nur der Origin als Referer versendet, und dies auch nur falls das Sicherheitslevel ident ist (es wird also beim Übergang von einer HTTPS auf eine HTTP Seite kein Referer gesetzt).
	\item[strict-origin-when-cross-origin]: es wird die volle URL innerhalb des identen Origin verwendet, wird auf Webseiten mit unterschiedlichen Origin zugegriffen (also z.B. beim Übergang auf externe Webseiten) wird nur der Origin als Referer verwendet. Falls die Sicherheit der Kommunikation schlechter wird (also z.B. beim Übergang von HTTPS auf HTTP) wird überhaupt kein Referer-Header versendet.
	\item[unsafe-url]: die gesamte URL wird im Referrer-Header immer übertragen.
\end{description}

Es wird empfohlen, eine \textit{Referrer-Policy} zu wählen, die nicht \textit{no-referrer-when-downgrade} bzw. \textit{unsafe-url} ist.

\section{Content-Security-Policy}
\index{Content-Security-Policy}
\index{HTTP Header!Content-Security-Policy}
\label{csp}

Die Content-Security-Policy (CSP) erlaubt es, eine umfangreiche Policy von Webservern an Browser zu übertragen. Prinzipiell sind mittels einer CSP viele der bereits erwähnten Security-Header abbildbar.

Die Content-Security-Policy kann entweder als HTTP-Header oder als Teil des HTML-Dokuments übertragen werden. Der verwendete HTTP Header heißt \textit{Content-Security-Policy}, bei Verwendung eines Meta-Tags würde der Code beispielsweise folgenderweise aussehen:

\begin{minted}{html}
<meta http-equiv="Content-Security-Policy" content="default-src 'self'; img-src https://*; child-src 'none';">
\end{minted}

Die grundlegende Funktionalität von CSP ist:

\begin{itemize}
	\item Definition von vertrauenswürdigem Javascript bzw. Javascript-Sourcen
	\item Sicherstellen, dass Seitenelemente (CSS, Bilder, etc.) nur aus vertrauenswürden Quellen bezogen werden.
	\item Definition der Interaktion mit externen Seiten (z.B. mittels iFrames)
	\item Sonstiges: Erhöhung der Verbindungssicherheit, etc.
\end{itemize}

\subsection{Trennung von HTML und JavaScript-Code}

Mittels CSP wird zumeist definiert aus welchen Quellen Javascript-Code geladen werden darf. Damit dadurch ein gutes Sicherheitsniveau erreicht werden kann, ist eine Trennung vom JavaScript-Code von den verwendeten HTML-Dateien notwendig. Falls Javascript innerhalb von HTML Dateien erlaubt ist, kann schwer unterschieden werden ob vorgefundener JavaScript-Code von der Applikation vorgesehen oder durch einen Angreifer eingeschleust worden ist.

Diese Trennung wird erreicht, wenn der gesamte JavaScript-Code in getrennten JS-Dateien bereitgestellt wird. Dieser wird in die HTML-Seite mittels einem \textit{script}-Tag eingebunden:

\begin{minted}{html}
<script src="externalfile.js"></script>
\end{minted}

In \textit{externalfile.js} wird nun der JavaScript-Code hinterlegt. Da a-priori keine Verbindung zwischen dem JavaScript-Code und dem HTML-File besteht, wird zumeist der \textit{\$(document).ready}-Callback verwendet. Code in diesem Callback wird ausgeführt, sobald das DOM fertig geladen wurde:

\begin{minted}{javascript}
$(document).ready(function()  {
	// binden des JavaScript-Codes an etwaige Elemente
	document.getElementById("btn").addEventListener('click', doSomething);

	// other Javascript code
});
\end{minted}

Bei diesem Beispiel wird nun durch die Methode \textit{addEventListener} die JavaScript-Methode \textit{doSomething} beim Klicken auf den Button mit der Id \textit{btn} aufgerufen.

\subsection{Verfügbare CSP-Elemente}

CSP besteht aus mehreren Direktiven, die Tabelle \ref{tbl:csp_elements} gibt eine kurze Übersicht der Möglichkeiten. Alle Direktiven die mit \textit{-src} enden erlauben die Verwendung ähnlicher Source-Werte, Tabelle \ref{tbl:csp_values} gibt ein paar Beispiele. Durch die Verwendung von ``unsafe-inline'' wird die Trennung zwischen HTML und JavaScript nicht mehr erzwungen und daher der Großteils des XSS-Schutzes neutralisiert. Diese Direktive sollte daher soweit wie möglich vermieden werden.


\begin{table}
	\begin{center}
\begin{tabular}{lp{10cm}}
	\toprule
	Name & Beschreibung\\ \midrule
	default-src & definiert die default Policy zum Laden von remote Elementen für die meisten Elemente. \\
	script-src & definiert vertrauenswürde Quellen für geladene JavaScript Dateien. \\
	style-src & definiert vertrauenswürde Quellen für geladene CSS-Dateien. \\
	plugin-types & object-src: definiert erlaubte Plugin Typen und deren vertrauenswürde Sourcen \\
	img-src, media-src, font-src & definiert vertrauenswürdige Quellen für die jeweiligen Dateitypen \\
	child-src (ehem. frame-src) &  definiert erlaubte Quellen für den Inhalt verwendeter Iframes. \\
	sandbox & aktiviert eine Sandbox für die aktuelle Resource ähnlich wie die Sandbox eines eingebetteten Iframas. Durch Optionen kann die Sandbox aufgeweicht werden. \\
	connect-src & Wird von JSONP, WebSockets und EventSource verwendet. \\
	form-action & welche URIs dürfen als Ziel eines Formulars dienen \\
	reflected-xss & entspricht X-XSS-Protection \\
	frame-ancestors & definiert, welche externen Seiten die Resource im Zuge eines Iframes verwenden dürfen, entspricht ca. einem \textit{X-Frame-Option}s. \\
	referrer & ähnlich wie Referrer-Header \\
	report-uri & CSP erlaubt die Angabe einer Reporting-URL. Im Fehlerfall wird an diese URL eine detaillierte Fehlermeldung reported. \\
	block-all-mixed-content & \\
	upgrade-insecure-requests & \\
	\bottomrule
\end{tabular}
	\caption{Häufig verwendete CSP-Direktiven}
	\label{tbl:csp_elements}
\end{center}
\end{table}

\begin{table}
	\begin{center}
	\begin{tabular}{lp{10cm}}
	\toprule
	Name & Beschreibung\\ \midrule
	* & erlaubt alle URIs ausgenommen \textit{data:}, \textit{blob:} und \textit{filesystem:} \\
	'none' & verbietet das Laden von Ressourcen. \\
	'self' & erlaubt das Laden von Ressourcen vom eigenen Origin \\
	data: & erlaubt das Bereitstellen von Ressourcen über data (base64-codierte Daten). \\
	domain & erlaubt das Laden von Daten von der entsprechenden Domain. Wildcards für Subdomains dürfen verwendet werden. Wird der Domainname mit \textit{https://} begonnen, muss HTTPS beim Zugriff verwendet werden \\
	https & erzwingt das Laden von Ressourcen über HTTPS, alle Domains sind erlaubt. \\
	'unsafe-inline' & erlaubt inline Javascript als auch \textit{javascript:} URIs \\
	'unsafe-eval' & erlaubt unsichere dynamische Code-Exekution mittels eval. \\
	'nonce-(nonce) & \textit{script}- oder \textit{style}-Tags werden exekutiert sofern diese ein nonce-Attribut besitzen welches ident zu dem Wert innerhalb des CSP ist. \\
	sha256-(hash) & Erlaubt die Exekution von Skripts sofern ihr Hash dem in der CSP angegeben Hash entsprechen. \\
	\bottomrule
\end{tabular}
	\caption{Häufig verwendete CSP Source-Werte}
	\label{tbl:csp_values}
\end{center}
\end{table}

\subsection{CSP-Nonces}

CSP-Nonces erlauben die Integration von CSP ohne die gesamte Web-Applikation nach dem Grundsatz der strikten Trennung von JavaScript und HTML entwickelt zu haben. Dieses System basiert darauf, dass innerhalb des CSP-Headers eine Nonce (zufällige einmalige Zahl) definiert wird und innerhalb der HTML Datei nur \textit{Script}- und \textit{Style}-Elemente nur exekutiert werden, wenn diese einen identen Wert als \textit{nonce}-Attribut gesetzt haben.

Wird zum Beispiel folgender CSP-Header verwendet:

\begin{minted}{text}
Content-Security-Policy: script-src 'nonce-2726c7f26c'
\end{minted}

würde das Skript nur exekutiert werden, wenn es die idente nonce (\textit{2726c7f26c}) innerhalb des \textit{script}-Tags verwendet:

\begin{minted}{html}
<script nonce="2726c7f26c">
  var inline = 1;
</script>
\end{minted}

Die Sicherheit dieses Verfahrens ist von zwei Annahmen abhängig:

\begin{enumerate}
	\item Der Angreifer darf die nonce nicht vorherbestimmten können. Ebenso muss bei jedem Seitenaufruf eine neue nonce generiert werden.
	\item Der Angreifer darf nicht die Möglichkeit besitzen, innerhalb eines sicheren Skript-Aufrufs (bei dem die richtige nonce gesetzt ist) bösartigen JavaScript-Code einzufügen.
\end{enumerate}

\subsection{CSP-Beispiele}

Ein einfaches Beispiel, welches das Laden von Ressourcen (Javascript, CSS, Images) nur vom eigenen Server erlaubt:

\begin{minted}{text}
Content-Security-Policy: default-src 'self'
\end{minted}

Folgendes Beispiel schränkt mögliche Angriffsvektoren bereits stark ein:

\begin{minted}{text}
Content-Security-Policy:
  object-src 'none';
  script-src 'nonce-{random}' 'unsafe-inline' 'strict-dynamic' https: http:;
  base-uri 'none';
\end{minted}

Folgende Einstellungen werden dadurch an den Browser übermittelt:

\begin{itemize}
	\item \textit{object-src: none} verhindert das Laden von Plugins wie z.B. Flash.
	\item Die verwendete script-src Line verwendet ``neues'' als auch ``altes'' CSP damit unterschiedliche Browserversionen sichere CSP Einstellungen bekommen. Die Kombination von \textit{nonce} und \textit{unsafe-inline} bewirkt bei neueren Browsern, dass \textit{script}-Tags nur verwendet werden, wenn die angegebene Nonce bei dem Skript-Tag als Attribut hinterlegt ist. Neuere Browser ignorieren in dem Fall \textit{unsafe-inline}. Ältere Browser ignorieren die Nonce, über \textit{unsafe-inline} werden allerdings ``normale'' script-Tags erlaubt. \textit{strict-dynamic} erlaubt das Laden von remote JavaScript-Dateien ausgehend von trusted Scripts. Moderne Browser ignorieren die zusätzlichen Schemas (http und https), während ältere Browser die kein \textit{strict-dynamic} erkennen durch die Schemas externe JavaScript-Dateien laden können.
	\item \textit{base-uri: none} deaktiviert das HTML \textit{base}-Element welches im Zusammenhang mit relativen Imports verwendet werden kann, um Injection-Angriffe durchzuführen.
\end{itemize}

\section{Reflektionsfragen}

\begin{enumerate}
	\item Was ist HTML Subresource Integrity (SRI), wie wird diese verwendet und gegen welche Angriffe schützt diese Maßnahme?
	\item Wie werden CSP verwendet? Gib ein Beispiel für CSPs? Wie funktionieren script-src nonces?
	\item Was ist die Referrer-Policy und warum sollte diese verwendet werden?
\end{enumerate}
